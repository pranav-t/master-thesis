\section{Fault Tolerance}
\label{sec:fault_tolerance}

Failure detection and recovery play a critical role, especially in
large-scale distributed systems.  In this section, we analyze the
behavior of VMS under VM and node crashes to ensure that after
appropriate recovery measures, views still converge.

\noindent
\textbf{VM crash} -- A VM maintains a queue with operations dispatched
to it. Upon a VM crash, these updates are lost, which may result in
non-converging views. Our recovery measures described here, ensure
view convergence under VM crash. For every update a VM propagates, it
stores the sequence ID of the processed operation in its commit log. A
VM crash triggers an event via Zookeeper
(cf. Section~\ref{sec:system_overview}), notifying the VMS
coordinator. Via the commit log, the coordinator retrieves the last
operation the crashed VM successfully propagated. The commit log order
is determined by the order the VM propagates updates, which results
from the stream of operations dispatched to it, respecting record
timeline semantics. Now, the coordinator executes the steps of
Algorithm~\ref{alg:crashed_vm}.



\begin{algorithm}
\caption{VM recovery in VMS}
\label{alg:crashed_vm}
\begin{algorithmic}[5]
\Procedure{$onVMCrashed$}{$nx, VM_{nx}, vm_c$}
\State $sendMessage(nx, withdraw, vm_c)$
\ForAll{$vm \in VM_{nx}$}	
\State $sendMessage(vm, requestSID)$
\EndFor
\ForAll{$vm \in VM_{nx}$}	
\State $SIDs \leftarrow receiveMessage(vm, SID)$	
\EndFor
\State $smallestSID \leftarrow minimum(SIDs)$
\State $sendMessage(nx, replay, smallestSeqID)$	
\EndProcedure
\end{algorithmic}
\end{algorithm}

First, the coordinator sends a withdraw command to the 
concerned NX component. The NX withdraws the crashed VM from the 
hash-ring and stops dispatching operations to it. This way no updates, 
that were in-fight while the VM crashed, are lost. Next, the coordinator 
obtains the last sequence ID of every VM (assigned to the node the 
crashed VM belonged to) and determines the smallest one, including the 
one from the crashed VM's commit log. The VMS advises the NX component 
to replay the log entries from the point of the smallest ID. This 
recovery measure is necessary for the following reasons: It might happen 
that a running view manager is executing a base table operation with a 
smaller sequence ID than the crashed one. Then, a replay from the crashed 
view manager's ID would cut off some of the client operations. Taking 
the smallest sequence ID, the VMS ensures that no operation gets lost.
However, there might be a case when the crashed view manager had already 
queued up operations but not written anything to the commit log.  Then 
the recovery of the sequence ID fails and again, operations may be lost.
The VM can just avoid this case by writing the first sequence ID it ever
receives from a node, directly to the commit log, marking it as 'not 
processed'. Then the transaction log can be replayed from this point. 
Another problem during replay is that some operations may be forwarded 
to the view managers again. This leads to duplicate view updates. 
But thanks to the signature mechanism (cf.Section~\ref{sec:consistency}), 
the duplicate operations can be identified and just ignored. 


\noindent
\textbf{Node crash} -- A node crash is handled by the recovery
mechanism of the KV-store. It selects a new node and replay's the
crashed node's TL. The coordinator of VMS just notices the
crash  and re-assigns the idle view managers to another
node.

