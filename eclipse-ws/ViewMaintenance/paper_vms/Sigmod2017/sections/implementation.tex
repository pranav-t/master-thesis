\section{Implementation}
\label{implementation}


In the following, we explain how we integrate the VMS with HBase. We 
need to conduct two steps. First, we integrate the NX component with the 
the HBase architecture. Second, we integrate the VM component with the 
HBase API. The components, shown in Figure~\ref{fig:system_overview} 
serve as a blueprint for the integration. 

\noindent
\textit{(i):} The nodes in HBase are called region servers. The region 
server writes the client updates to the write-ahead log (i.e. the 
transaction log). Now, the NX component needs to access the write-ahead 
log and retrieve the operations from it. As stated above HBase uses HDFS 
as underlying file system. All of the database files are stored here, 
including the write-ahead log. We can use the HDFS API to directly 
access a file and read from it sequentially. However, the data objects 
in the write-ahead logs are stored in a serialized and compressed form. 
Thus, we have to use an internal class of HBase called 
\texttt{HLog.Reader}. The reader class iterates over the log and 
retrieves the entries in a readable form, as described in 
Section~\ref{sec:kv_model}. This form is used by the NX component to 
extract the row key and route the operation to the appropriate view 
manager. We could extend different KV-store implementations. The 
KV-store model usually includes a operation log, used for recovery or 
replication reasons. The challenge here is to deserialize or decompress 
the data. Once, the operations are received they can be applied to our 
data model and processed by the NX components. Note, that an instance of 
the NX component is always deployed together with one node. 

\noindent
\textit{(ii):}As the view manager starts processing the operations and 
materializing the views, it connects to the KV-store API. The view 
manager uses an abstract class \texttt{KVSTableService}, which contains 
all of the methods, that are required to access the KV-store and build 
up the materialized views. These methods need to be implemented: 
\texttt{put()}, \texttt{get()}, \texttt{delete()}, 
\texttt{checkAndPut()} and \texttt{checkAndDelete()}. The view manager 
is then able to access HBase and update the view tables. We could 
implement the class to access every type of database. The view manager 
could also be adapted to persist and update the materialized views in a 
SQL-based storage. 

