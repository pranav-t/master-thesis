\subsection{Proof of Consistency}


Theorem~\ref{theo:strong_consistency} states that a VMS system fulfilling all 
three requirements achieves at least strong consistency. In the following, we 
will present a proof for this theorem, which is organized in three stages: we 
start with proving convergence and then present extensions to also prove weak 
consistency  and finally strong consistency.





\subsubsection{Notation}
First, we define the following notation for keys, 
operations on keys, and the ordering of operations. Let $k_x$ denote a key in 
a base table, where $x \in X = \{1, \dots, m\}$, and $X$ is the 
table's key range. Further, let an operation on key $k_x$ be defined as 
$t[k_x]$, and a totally ordered sequence of such operations be denoted by 
$\langle t_1, t_2, t_{3}, \dots, t_N\rangle$, where $N$ defines the total 
number of operations on the table in a given timespan. 
Hence, a generalized sequence of operations on a base table is 
represented by $\langle t_1[k_{x_1}], t_2[k_{x_2}], \dots, 
t_n[k_{x_m}]\rangle$, where $k_{x_1}, \dots, k_{x_m} \in X$ can be arbitrarily 
chosen from the base table's key range for every timestamp. Based on this 
generalized sequence every other sequence of operations can be derived, e.g. 
$\langle t_1[k_{1}],t_2[k_{2}],t_3[k_{1}]\rangle$. In some cases, we also want 
to keep the ordering of operations variable. For that reason, we introduce an 
index $s_i$ with $i \in \{1, \dots, n\}$. Then we can write the arbitrary 
sequence as $\langle t_{s_1}[k_{x_1}], t_{s_2}[k_{x_2}], \dots, 
t_{s_n}[k_{x_n}]\rangle$ with $s_1\neq \dots \neq s_n$. Using this notion, we 
are capable of representing every possible sequence of update operations in 
the system. 

The index $t^{(i)}[k_x]$ is used to express a sequence of operations on 
a single row key (i.e. the time-line). For example, a sequence of 
operations on row key $k_x$ is denoted as $\langle 
t^{(1)}[k_x],t^{(2)}[k_x]...,t^{( \omega)}[k_x]\rangle$. The last 
operation on a particular row key is always denoted with $\omega$. (see 
Section~\ref{sec:consistency}).

For the proof, we assume such an arbitrary sequence of base table operations
 and then we show that --- given the requirements in the theorem --- a VMS 
 system will produce correct view results. Formally, we show that $V_f=View(B_f)$. 

\subsubsection{Convergence}
\label{sub:proof_convergence}
As mentioned earlier, every view type defines its own mapping from base table 
to view table records. Thus, we prove the different cases separately. 

\noindent
\textit{One-to-one mapping:} \texttt{SELECTION}, \texttt{PROJECTION} 
views define a one-to-one mapping between base and view table. The row 
key of the base table is also the row key of the view table. Operations 
for both view types are idempotent, meaning an operation could be 
applied several times without changing the result. The view record is 
always a representation of the last base table operation applied. A 
correct view record with row key $k$ is defined as the last operation on 
the row key in the base table key, e.g. $k\leftarrow 
View(t^{(\omega)}[k])$. The function $View$ calculates the view record 
(by using the appropriate view definition) and applies the result to the 
correct view key. A view table converges, if all view records are 
computed correctly in the last view state. 

We start defining an arbitrary sequence of operations on the base table, 
shown in Step~\ref{proof:oo_step1}. Clients can update different row 
keys in the base table, using put or delete operations. Likewise, they 
can update the same row key multiple times. The update operations of the 
clients form a particular global order, expressed through $t_1..t_n$. 
They take the base table from its initial state $B_0$ to its final state 
$B_f$. In the next step, all update operations get forwarded, causing 
the ordering of operations to be lost. If we would continue working with 
an unordered set, convergence of the view will be violated at some 
point. For this reason, we apply requirement (iii) (,i.e. the time-line 
requirement) to our equation, as depicted in Step~\ref{proof:oo_step3}. 
As updates operations do not influence each other (see requirement 
(ii)), we are able to create a set of sub-sequences. These 
sub-sequences only consist of updates that have been applied to the same 
row key. Likewise the sub-sequences contain all operations from the 
previous step (see requirement (i)). 
%
\begin{subequations}
  \begin{align}
 S_b&=\langle t_1[k_{x_1}],....,t_n[k_{x_n}] \rangle;\;\;\Big(B_0 \overset{S_b}{\rightarrow}B_f\Big) \label{proof:oo_step1}\\ 
 S_1&=\langle t^{(1)}[k_{1}],..,t^{(\omega_1)}[k_{1}]\rangle,..,S_m=\langle t^{(1)}[k_{m}],..,t^{(\omega_m)}[k_{m}]\rangle\label{proof:oo_step3}\\
  S_1&=\langle t^{(\omega_1)}[k_{1}]\rangle,..,S_m=\langle t^{(\omega_m)}[k_{m}]\rangle\label{proof:oo_step4}\\
 S_v&=\langle t_{s_1}^{(\omega_1)}[k_{1}],..t_{s_n}^{(\omega_m)}[k_{m}]\rangle;\; \Big(V_0\overset{S_v}{\rightarrow}V_f\Big)\label{proof:oo_step5}\\
 	V_f&=k_x\leftarrow View(t^{(\omega_x)}[k_x])=View(B_f)\label{proof:oo_step6}
  \end{align}
\end{subequations}
%
In Step~\ref{proof:oo_step4}, we pick the last element of all 
sub-sequences and eliminate the rest. As stated above, only the last 
operation on a particular row key has an influence on the final view 
state. Again, we observe that the time-line of a row key is vital. If it 
is broken, e.g. for row key $k_1$, then an operation 
$t^{(\omega-1)}[k_{1}]$ can be incorporated into the final result and 
render it incorrect. After the elimination, we unite the operations 
again in Step~\ref{proof:oo_step5}. The reunion allows the remaining 
operations to be executed in every possible order (i.e. every operation 
could be executed in parallel). Finally, the view definition is applied 
to every operation in $R$. This leads to the correct final view records 
and hence, to convergence. The \texttt{DELTA} view also defines a 
one-to-one mapping between base and view records. In contrast to the 
aforementioned views, the results of the \texttt{DELTA} view do not only 
relate to the last, but to the two last operations. Therefore, we need 
to change the last three steps of the proof as follows: 
%
\begin{subequations}
  \begin{align}
  S_1&=\langle t^{(\omega_1-1)}[k_{1}],t^{(\omega_1)}[k_{1}]\rangle,..,S_m=\langle t^{(\omega_m-1)}[k_{m}],t^{(\omega_m)}[k_{m}]\rangle\\
   S_v&=\langle t_{s_1}^{(\omega_1-1)}[k_{1}],t_{s_2}^{(\omega_1)}[k_{1}],..t_{s_{n-1}}^{(\omega_m-1)}[k_{m}],t_{s_n}^{(\omega_m)}[k_{m}]\rangle\label{proof:ood_step2}\\
   &(\forall t_{s_i}^{(\omega_y-1)}\in S_v)(\exists t_{s_j}^{(\omega_y)}\in S_v)\;s_i < s_j;\;\Big(V_0\overset{S_v}{\rightarrow}V_f\Big)\notag\\
 	V_f&=k_x\leftarrow View(t^{(\omega_x-1)}[k_x],t^{(\omega_x)}[k_x])=View(B_f)
  \end{align}
\end{subequations}
%
As can be observed in Step~\ref{proof:ood_step2}, the two last 
operations of a time-line are included into the final result. However, 
the sequence can be arbitrarily ordered, but needs to preserve the 
time-line of both last operations (i.e. $\omega-1$ must always precede 
$\omega$). Computing $V_f$ leads to the valid final state and the 
\texttt{DELTA} view converges. 

\noindent
\textit{Many-to-one mapping:} (\texttt{PRE}-)\texttt{AGGREGATION} and 
\texttt{INDEX} views define a many-to-one mapping between base and view 
table. The row key of the view table is the aggregation key. Multiple 
row keys in the base table can relate to a particular aggregation key. 
However, a base table row has always only one aggregation key. A correct 
view record with aggregation key $x$ is defined as the combination of 
multiple base records $k_{x_1}..k_{x_j}$, related to the particular key. 
In terms of incremental view maintenance, the correct view record can be 
defined as a number of last operations, that have been applied to this 
combination of base records: $x \leftarrow 
View(t^{(\omega_1)}[k_{x_1}],..,t^{(\omega_j)}[k_{x_j}])$. In case of a 
\texttt{SUM} view e.g., this resolves to $x \leftarrow 
f(t^{(\omega_1)}[k_1])+..+f(t^{(\omega_j)}[k_j])$. We start again, 
defining an arbitrary sequence of base table operations in 
Step~\ref{proof:mo_step1}. In contrast to the previously handled views, 
we are now processing $\delta$-operations. We construct a number of $m$ 
sub- sequences, each containing the $\delta$-operations of one 
particular base record key. In Step~\ref{proof:mo_step2}, we merge the 
$\delta$-operations together. All $\delta$- operations add up to form 
the last transaction as the end result (i.e. $\delta(t^{(1)}[k_{1}])+
..+\delta(t^{( \omega_1)}[k_{1}])=t^{(\omega_1)}[k_{1}]$). 
%\begin{subequations}
%  \begin{align}
%  \{t_1(k_1),..,t_n(k_1)..t_1(k_n),..,t_n(k_n)\}\\
% \{\langle t_1(k_1),..,t_n(k_1)\rangle,..\langle t_1(k_n),..,t_n(k_n)\rangle\}\\
% \{\langle t_n(k_1)\rangle,..\langle t_n(k_n)\rangle\}\\
% 	x=f(t_n(k_1))+..+f(t_n(k_n))
%  \end{align}
%\end{subequations}
%
\begin{subequations}
  \begin{align}
  S_b&=\langle t_1[k_{x_1}],....,t_n[k_{x_n}] \rangle;\;\Big(B_0 \overset{S_b}{\rightarrow}B_f\Big) \label{proof:mo_step1}\\ 
 S_1&=\langle \delta(t^{(1)}[k_{1}]),.,\delta(t^{(\omega_1)}[k_{1}])\rangle,..,\label{proof:mo_step2}\\
 &\hspace{10 mm}S_m=\langle \delta(t^{(1)}[k_{m}]),.,\delta(t^{(\omega_m)}[k_{m}])\rangle\notag\\
  S_1&=\langle t^{(\omega_1)}[k_{1}]\rangle,..,S_m=\langle t^{(\omega_m)}[k_{m}]\rangle\\
 S_v&=\langle t_{s_1}^{(\omega_1)}[k_{1}],..t_{s_n}^{(\omega_m)}[k_{m}]\rangle;\;\Big(V_0\overset{S_v}{\rightarrow}V_f\Big)\label{proof:mo_step4}\\
 	V_f&=x\leftarrow View(t^{(\omega_{x_1})}[k_{x_1}],.., t^{(\omega_{x_j})}[k_{x_j}])=View(B_f)\label{proof:mo_step5}
 	%&\hspace{10 mm}x_1,..,x_j \in \{1,..,m\};\;x_1\neq,..,\neq x_j \notag
  \end{align}
\end{subequations}
%
Now, we can unite the single sequences as done before. We retrieve a 
final sequence as shown in Step~\ref{proof:mo_step4}. The operations of 
this sequence are then applied to the view --- simultaneously they are 
grouped and stored according to their aggregation key. The final view 
records are calculated correctly, as depicted in 
Step~\ref{proof:mo_step5}, which causes the aggregation view to 
converge. 

\textit{Many-to-many mapping:} (\texttt{REVERSE}-)\texttt{JOIN} views 
define a many-to-many mapping between base and view table. The row key 
of the view table is a composite key of both join tables' row key. 
Multiple records of both base tables form a set of multiple view records 
in the view table. Since the joining of tables takes place in the 
\texttt{REVERSE JOIN} view, we prove convergence only for this view 
type. A \texttt{REVERSE JOIN} view has a structure that is similar to an 
aggregation view. The row key of the \texttt{REVERSE JOIN} view is the 
join key of both tables. All base table records are grouped according to 
this join key. But in contrast to an aggregation view the 
\texttt{REVERSE JOIN} view combines two base tables to create one view 
table. A correct view record with join key $x$ is defined as a 
combination of operations on keys $k_1..k_n$ from join table $A$ and 
operations on keys $l_1..l_p$ from join table $B$. In order to represent 
both keys we introduce an additional variable $z_1,..,z_n \in 
\{k_1,..,k_m, l_1,..,l_p\}$. Then, the correct view record is defined 
as: $x \leftarrow View(t^{(\omega_1)}[z_1], ..,t^{(\omega_j)}[z_j])$. 
We start with a sequence of arbitrary client updates to both base 
tables, as depicted in Step~\ref{proof:mm_step1}. Then, the order of 
updates is lost and the time-line requirement is realized in 
Step~\ref{proof:mm_step2}. 
%
\begin{subequations}
  \begin{align}
  S_b&=\langle t_1[z_1],..,t_n[z_n]\rangle;\;\Big(B_0 \overset{S_b}{\rightarrow}B_f\Big)\label{proof:mm_step1}\\ 
 S_1&=\langle \delta(t^{(1)}[k_{1}]),..,\delta(t^{(\omega_1)}[k_{1}])\rangle,..,S_m,..,S_{m+1},..\label{proof:mm_step2}\\ 
&\hspace{10 mm}S_{m+p}=\langle \delta(t^{(1)}[l_{p}]),..,\delta(t^{(\omega_p)}[l_{p}])\rangle\notag\\
  S_1&=\langle t^{(\omega_{k_1})}[k_{1}]\rangle,.., S_m=\langle t^{(\omega_{k_m})}[k_{m}]\rangle,..,\\
 &\hspace{10 mm}S_{m+1}=\langle t^{(\omega_{l_1})}[l_{1}]\rangle,..,S_{m+p}=\langle t^{(\omega_{l_p})}[l_{p}]\rangle\notag\\
 S_v&=\langle t^{(\omega_{k_1})}[k_{1}],..t^{(\omega_{k_m})}[k_{m}],\\
 &\hspace{10 mm}t^{(\omega_{l_1})}[l_{1}],..,t^{(\omega_{l_p})}[l_{p}] \rangle;\;\Big(V_0\overset{S_v}{\rightarrow}V_f\Big)\label{proof:mm_step3}\\
 	V_f&=x\leftarrow View(t^{(\omega_{z_1})}[z_1],.., t^{(\omega_{z_j})}[z_j])=View(B_f)\label{proof:mm_step4}
 	%&\hspace{10 mm}z_1,..,z_j \in \{k_1,..,k_m,l_1,..,l_p\}; z_1\neq,..,\neq z_j ;\; \notag
  \end{align}
\end{subequations}
%
We eliminate all but the last operations $\omega$ and reunite the 
operations in Step~\ref{proof:mm_step3}. This leads to the final 
Step~\ref{proof:mm_step4}, where the operations are applied to the view 
record. Since the view records are calculated correctly (i.e. only the 
last operations of the row keys are included) we conclude that the view 
converges. 

\subsubsection{Weak consistency} 
\label{sub:proof_weak}

Weak consistency has been defined as follows: Weak consistency is given 
if the view converges and all intermediary view states are valid, 
meaning they can be derived from one of the base states with 
$V_j=View(B_i)$. As we already proved convergence, we need show that all 
the intermediary view states are correct likewise. We start again with 
an arbitrary sequence of operations in Step~\ref{proofw:oo_step1}. In 
order to generate an intermediate base state, we cut the sequence at any 
point before an operation $t_a$, with $1 < a < n$. After the ordering is 
lost, we apply the time-line consistency. But in contrast to before, we 
are not capable of processing the complete time-line (i.e. 
$(1)..(\omega)$). Instead, we process the time-line until an 
intermediary element $\alpha_x \leq \omega_x$. 
%
\begin{subequations}
  \begin{align}
  S_b&=\langle t_1[k_{x_1}],....,t_n[k_{x_n}] \rangle;\;\Big(B_0 \overset{S_b}{\rightarrow}B_{a}\Big)\label{proofw:oo_step1}\\
 S_1&=\langle t^{(1)}[k_{1}],..,t^{(\alpha_1)}[k_{1}]\rangle,..,S_m=\langle t^{(1)}[k_{m}],..,t^{(\alpha_m)}[k_{m}]\rangle\\
 S_v&=\langle t_{s_1}^{(1)}[k_{1}],..t_{s_i}^{(\alpha_1)}[k_{1}],..,t_{s_j}^{(1)}[k_{m}],..,t_{s_a}^{(\alpha_m)}[k_{m}]\rangle;\;\Big(V_0\overset{S_v}{\rightarrow}V_a\Big)\\
 (\forall& t_{s_1}^{(\lambda_1)}[k_{x_1}] \in S_v)(\forall t_{s_2}^{(\lambda_2)}[k_{x_2}] \in S_v):(x_1=x_2)\;\land\;(\lambda_1<\lambda_2) \Rightarrow s_1 < s_2\notag\\
V_{a}&=k_x\leftarrow View(t^{(\alpha_x)}[k_x])=View(B_{\alpha})
  \end{align}
\end{subequations}


\subsubsection{Strong consistency}
\label{sub:proof_strong}

Strong consistency has been defined as follows: Weak consistency is 
achieved and the following conditions hold true. All pairs of view 
states $V_i$ and $V_j$ that are in a relation $V_i \leq V_j$ are derived 
from base states $Bi$ and $B_j$ that are also in a relation $B_i \leq 
B_j$. Since weak consistency is already proven, we only need to prove 
the statement $V_i \leq V_j \Rightarrow B_i \leq B_j$. If this statement 
is negated, then only two of the following cases can occur: Either $V_i 
\leq V_j \Rightarrow B_i \parallel B_j$ or $V_i \leq V_j \Rightarrow B_i 
\geq B_j$. Both cases can only be constructed by breaking the record 
time-line. To be precise: At least one record has to exists, whose 
time-line is broken. Formally, we demand $ (\exists t_l \in B_i)(\forall 
t_k \in B_j):(r(t_l)=r(t_k))\land(l > k) $. Because requirement (iv) 
prevents the breaking of time-lines, we conclude that both cases are not 
possible. Thus, we have proven strong consistency by contradiction. 
