\section{Consistency models}
\label{sec:consistency_models}

The consistency model we use, is derived from earlier consistency
models that have been developed to measure the degree of consistency
during view maintenance. Now, we describe formally how these models build on
top of each other.

\textbf{1:1-model:} The very early models contain nothing but a base 
table on one node (i.e. a standalone database) and a view table on 
another node (i.e. the data warehouse) \cite{zhuge:view}. The base table 
is denoted as $B$ and the view table as $V$. Once operations $t_1..t_n 
\in T$ are applied to the base table $B$, the base table changes its 
state. This leads to different \textit{base states} that are denoted as 
$B_0..B_i..B_f$. The initial state of the base table is $B_0$. Every 
operation on the base table produces a new base state until the final 
state $B_f$ is reached. The operator $\leq$ is used to express the 
relation between states, e.g. $B_i \leq B_j$. A state $B_j$ is said to 
be greater than a state $B_i$ if $B_j$ contains all operations of $B_i$ 
plus a number of subsequent operations. Formally, we demand $(\forall t 
\in B_i)\;t \in B_j$ and $(\exists t_k \in B_j)(\forall t_l \in B_i)\;k 
> l $. The view table in the data warehouse also consists of a series of 
\textit{view states} $V_0..V_j..V_f$. Consistency is analysed by 
comparing the system's view states with the theoretical correct view 
states, derived from the appropriate base state. We derive the correct 
view state by applying function $View(B)$. Thus, we say a view state 
corresponds to a base state when $V_j=View(B_i)$. 


\textbf{1:N model:} In a next step, the consistency model was extended 
to describe view maintenance in a multi-source data 
warehouse\cite{zhuge:strobe}. Here, the data warehouse is still located 
on one node, but the base table information is consumed from $n$ 
different source nodes. Operations on a single source are serializable 
and executed in sequence. However, operations from different sources 
doesn't need to preserve ordering. The base table states are defined as 
$B_0..B_i..B_f$. The definition $\leq$ can be applied between the 
produced states, again. But, in order to match the relaxed ordering 
requirements, we have to extend the declaration from before to $(\exists 
t_k \in B_j)(\forall t_l \in B_i)\;k > l\;\land\;s(t_k)=s(t_l)$. 
Function $s()$ delivers, when applied to an operation, the source from 
which the operation originated. However, two operations on different 
sources are processed in parallel, allowing two possible intermediary 
base states $B_i$ and $B_j$. $B_i$ does not contain all operations of 
$B_j$ and vice versa. Thus, $B_i$ and $B_j$ are neither $\leq$ nor 
$\geq$. We denote this relation as $B_i \parallel B_j$. As already 
stated, this extended model represents a relaxation to the original 
model. We do not claim global serializability of operations any more. 


\textbf{N:M model:} In our context, we stream operations from multiple 
sources (i.e. nodes) and apply them to different parts of the view 
table, located on different nodes. For we want to parallelize view 
maintenance and improve performance to a great extent, we further relax 
the consistency model. We do not claim the stream of a local source to 
be in sequence. Instead, we define consistency on a record base level. 
Every sequence of operations, that is applied to a specific row key has 
to be in sequence. A base state $B_j$ is said to be greater than $B_i$ 
if all operations to a specific row key in $B_i$ are included in $B_j$ 
plus a number of subsequent operations. Formally we write $(\exists t_k 
\in B_j)(\forall t_l \in B_i)\;k > l\;\land\; r(t_k)=r(t_l)$. Function 
$r()$ delivers, when applied to an operation, the row key of that 
operation. Now, we define consistency levels in agreement to the 
N:M-model. Formally, the consistency levels of the N:M-model are 
defined as follows:


\begin{enumerate}
\item \textit{Convergence}: A view table converges, if after the system
  quiesces, the last view state $V_f$ is computed correctly. This
  means it corresponds to the evaluation of the view expression over
  the final base state $V_f=View(B_f)$. 

\item \textit{Weak consistency} Weak consistency is given if the view
  converges and all intermediate view states are valid, meaning that
  there exists a base table state in the sequence of base table states
  produced by the update sequence from which they can be derived
  %$V_j=View(B_i)$.

\item \textit{Strong consistency}: Weak consistency is achieved and the
  following condition is true. All pairs of view states $V_i$ and
  $V_j$ that are in a relation $V_i \leq V_j$ are derived from base
  states $B_i$ and $B_j$ that are also in a relation $B_i \leq B_j$.

\item \textit{Complete consistency}: Strong consistency is achieved
  and every base state $B_i$ of a valid base state sequence is
  reflected in a view state $V_i$. Valid base state sequence means
  $B_0 \leq B_i \leq B_{i+1}\leq B_f$.
\end{enumerate}


