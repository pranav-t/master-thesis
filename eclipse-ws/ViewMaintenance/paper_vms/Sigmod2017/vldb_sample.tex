% THIS IS AN EXAMPLE DOCUMENT FOR VLDB 2012
% based on ACM SIGPROC-SP.TEX VERSION 2.7
% Modified by  Gerald Weber <gerald@cs.auckland.ac.nz>
% Removed the requirement to include *bbl file in here. (AhmetSacan, Sep2012)
% Fixed the equation on page 3 to prevent line overflow. (AhmetSacan, Sep2012)

\documentclass{vldb}
\usepackage{balance}  % for  \balance command ON LAST PAGE  (only there!)
\usepackage{graphicx}
\usepackage{cite}
\usepackage{caption}
\usepackage{subcaption}
\usepackage{alltt}
\usepackage[hidelinks]{hyperref}
\usepackage{algpseudocode}
\usepackage{algorithm}
\bibliographystyle{unsrt}
\usepackage{amssymb}
\usepackage[subtle]{savetrees}
\usepackage{amsmath}
\usepackage{booktabs}
\usepackage[table]{xcolor}
\usepackage{array}
\usepackage{pgfplots}
\usepackage{listings}     
\pgfplotsset{compat=1.13}


\newcommand{\KVS}{{\small \textsf{KV-store}}}
\newcommand{\KVSs}{{\small \textsf{KV-stores}}}
\newcommand{\HB}{{\small \textsf{HBase}}}
\newcommand{\TL}{\textsf{TL}}
\newcommand{\BT}{{\small \textsf{Bigtable}}}
\newcommand{\CAS}{{\small \textsf{Cassandra}}}
\newcommand{\DY}{{\small \textsf{Dynamo}}}
\newcommand{\PN}{{\small \textsf{PNUTS}}}
\newcommand{\VMS}{{\small \textsf{VMS}}}
\newcommand{\VMs}{{\small \textsf{VMs}}}
\newcommand{\VM}{{\small \textsf{VM}}}

\begin{document}

% ****************** TITLE ****************************************

\title{Dynamic Scalable View Maintenance in KV-stores}

% possible, but not really needed or used for PVLDB:
%\subtitle{[Extended Abstract]
%\titlenote{A full version of this paper is available as\textit{Author's Guide to Preparing ACM SIG Proceedings Using \LaTeX$2_\epsilon$\ and BibTeX} at \texttt{www.acm.org/eaddress.htm}}}

% ****************** AUTHORS **************************************

% You need the command \numberofauthors to handle the 'placement
% and alignment' of the authors beneath the title.
%
% For aesthetic reasons, we recommend 'three authors at a time'
% i.e. three 'name/affiliation blocks' be placed beneath the title.
%
% NOTE: You are NOT restricted in how many 'rows' of
% "name/affiliations" may appear. We just ask that you restrict
% the number of 'columns' to three.
%
% Because of the available 'opening page real-estate'
% we ask you to refrain from putting more than six authors
% (two rows with three columns) beneath the article title.
% More than six makes the first-page appear very cluttered indeed.
%
% Use the \alignauthor commands to handle the names
% and affiliations for an 'aesthetic maximum' of six authors.
% Add names, affiliations, addresses for
% the seventh etc. author(s) as the argument for the
% \additionalauthors command.
% These 'additional authors' will be output/set for you
% without further effort on your part as the last section in
% the body of your article BEFORE References or any Appendices.

\numberofauthors{3} %  in this sample file, there are a *total*
% of EIGHT authors. SIX appear on the 'first-page' (for formatting
% reasons) and the remaining two appear in the \additionalauthors section.

\author{
{\rm Jan Adler}\\
TU M\"unchen
\and
{\rm Martin Jergler}\\
TU M\"unchen
\and
{\rm Arno Jacobsen}\\
TU M\"unchen
}
% There's nothing stopping you putting the seventh, eighth, etc.
% author on the opening page (as the 'third row') but we ask,
% for aesthetic reasons that you place these 'additional authors'
% in the \additional authors block, viz.
\additionalauthors{Additional authors: John Smith (The Th{\o}rv\"{a}ld Group, {\texttt{jsmith@affiliation.org}}), Julius P.~Kumquat
(The \raggedright{Kumquat} Consortium, {\small \texttt{jpkumquat@consortium.net}}), and Ahmet Sacan (Drexel University, {\small \texttt{ahmetdevel@gmail.com}})}
\date{30 July 1999}
% Just remember to make sure that the TOTAL number of authors
% is the number that will appear on the first page PLUS the
% number that will appear in the \additionalauthors section.


\maketitle

\begin{abstract}
Distributed \textit{key-value stores} have become the solution of
choice for many data-intensive applications. However, their limited
query language support imposes challenges for applications that
require sophisticated query capabilities.  To address this problem, in
this paper, we develop the \textit{View Maintenance System} (\VMS) to
incrementally maintain selection, projection, aggregation, and join
views on behalf of applications.  We design \VMS\ given a generic
\KVS\ model based on a small number of features available in many
popular store architectures.  \VMS\ supports the maintenance of
hundreds of views in parallel, while simultaneously providing
guarantees for view consistency, even under node crash scenarios.  To
evaluate our concepts, we deliver a full-fledged implementation of
\VMS\ through Apache's \HB\ and conduct an extensive experimental
study. Exploiting parallel maintenance, \VMS\ achieve 
throughputs up to 60k view updates per second (120k table updates). 
\end{abstract}



\section{Introduction}


KV-stores
Nowadays, distributed KV-stores are powerful databases to handle large amounts
of data. They spread data over the network, thereby allowing to scale the system
horizontally. In case of scarce resources or heavy amounts of client request, 
the administrator can just add more nodes to relive the system. Spreading the data
also minimizes access time to the clients and improves availability. A KV-Store is 
optimized for high write throughput and -- due to its distributed nature -- can 
serve thousands of client requests at a time. Since data is spread, computations 
can be executed in parallel on multiple parts of the data sets. Likewise, the 
distributed KV-store replicates data over multiple nodes and provides automated 
mechanisms for fault tolerance. 

KV-store weakness
Offering all the mentioned features the KV-store avoids a complex data model and 
sacrifices a powerful query language for a much simpler API (comprising of put, get
and delete operations). Tables are schema-less, records are identified by a
row key, data is stored into byte arrays. To perform analytical queries on tables, 
we perform scans over large data sets. This implies large run times for the queries 
and leads to performance gaps, during which clients suffer slow response times. 
Likewise, the logic for the SQL query is executed on application side. Often times 
code is application specific and cannot be reused later. 

Known approaches 
One solution to the problem is loading the table snapshots of KV-store into an 
external data warehouse. Here, all kinds of analytical operations can be applied to
the data set (in-memory technologies can be used. Since loading a complete data 
set is a long running task and we want to provide up-to-date results, this it not
an option. 




VMS


Our approach
Materialized views are a well known concept to obtain results from a database
and cache them for subsequent client requests. A materialized view can
express any SQL-query and provides fast access to the query results. However,
materialized views introduce the problem of view maintenance -- once the base
data changes the view needs to be updated. To update the views efficiently,
we apply the principles of deferred and incremental view maintenance. We just 
update those part of the view that are affected by the corresponding base 
table update. 


In paper XY, we designed a View Maintenance System that performs
view maintenance at scale and tackles all the related consistency issues. Now,
we use the View Maintenance System to implement all kinds of basic view types
(e.g. selection, aggregation, join, etc..). Then, we provide a SQL layer to
translate SQL queries into a view maintenance plan. We use a DAG (Directed acyclic 
graphs) to describe the query as a tree of connected materialized views (of basic 
view types). We describe the DAG of a general SQL pattern to match any kind 
of SQL query. Using VMS, we create and maintain the view tree in a hierarchical
manner and provide the latest results of the query. Since each materialized view 
adds maintenance effort, we introduce methods to optimize (i.e. merge intermediate
views, reorder the DAG) and reduce storage, as well as computation cost.


 
The paper is structured as follows: first, we define a general data model for 
KV-stores by looking at the different
available implementations.Then we briefly discuss the VMS designed in paper 
XY. In the core part of the paper, in Section view types, we define the basic view
types that can be combined to form higher level abstractions. In Section 4, we
show how a view maintenance plan can be derived from any SQL query. We also
optimize this maintenance plan to reduce storage and computation cost. Finally,
we evaluate our approach in an extensive experimental study, using HBase as a
KV-store implementation.






%
\section{Background}
In this section, we explain the infrastructure, necessary to manage base
and view tables. We start with a general description of the KV-store and 
its data model. Further, we explain the View Maintenance System that 
creates and updates the materialized views (stored in KV-store). 

\subsection{KV store}

Some KV-stores are based on a master-slave architecture, i.e. HBase;
other KV-stores run without a master, i.e. Cassandra (a leader 
is elected to perform tasks controlling the KV-store). 
In both cases a \textit{node} represents the unit of scalability -- 
as arbitrary instances can be spawned in the network (see 
Figure~\ref{fig:kv_model}). The node persists the actual data.  But in
contrast to a traditional SQL-database, a node manages only part of the 
overall data (and request load). As load grows in the KV-store, more 
nodes can be added to the system; likewise, nodes can be removed as load 
declines. The KV-store will automatically adapt to the new situation and 
integrate, respectively drop the resource. KV-stores differ in how they 
accommodate;  they also differ in how they perform load balancing and 
recovery (in face of node crashes). However, with regard to nodes, we can 
describe a set of universal events that occur in every KV-store 
(cf. Table~\ref{tab:kvs_a_events}).

A \textit{table} in a KV-store does not follow a fixed schema. It stores 
a set of table records called \textit{rows}. A row is uniquely identified by a
\textit{row key}. A row holds a variable number of \textit{columns}
(i.e., a set of column-value pairs). Columns can be further
grouped into \textit{column families}. Column families provide fast
sequential access to a subset of columns. They are determined when a
table is created and affect the way the KV-store organizes its table
files.\\ 
\textit{Key ranges} serve to partition a table into multiple
parts that can be distribute over multiple nodes.  Key ranges are
defined as an interval with a start and an end row key.  PNUTS refers
to this partitioning mechanisms as tablets, while HBase
refers to key ranges as regions. Multiple regions can be
assigned to a node, often referred to as a region server. 
In general, a KV-store can split and move key ranges between nodes to 
balance system load or to achieve a uniform distribution of data. With
regard to key ranges, we can also describe a set of universal events 
(cf. Table~\ref{tab:kvs_a_events}).


The data model of a KV-store differs from that of a
relational DBMS.  We describe a model that is representative for
today's KV-stores. The model serves throughout the paper to help
specify views and view update programs. Typically, KV-stores do not
required fixed data schemas, but rather accommodate dynamic schema
changes.

Thus, we formalize the data model of a KV-store as a map of key-value 
pairs $\{\langle k_1, v_1\rangle,..,\langle k_n,v_n\rangle\}$ described 
by a function $f:K \rightarrow V$. Systems like BigTable, HBase and 
Cassandra established data models that are multi-dimensional maps 
storing a row together with a variable number of columns per row. For 
example, the 2-dimensional case creates a structure 
$\{\langle(k_1,c_1),v_{1,1}\rangle,\langle (k_n,c_n) ,v_{n,n}\rangle\}$ 
with a composite key $(k_n,c_n)$ that maps to a value $v_{n,n}$ 
described by $f:(K,C)\rightarrow V$. In the 3-dimensional case, another 
parameter, a timestamp, for example, is added to the key, which may 
serve versioning purposes. For the intentions in this paper, the 
2-dimensional model suffices.\footnote{Our approach also works with the 
1-dimensional case, which is representative for simple key-blob stores. 
We use the 2-dimensional case here, as it is more expressive.} 


We denote a table by $A = (K, F)$, where $K$ represents the row key and 
$F$ a \textit{column family}. Column families are defined when a table 
is created. They are used in practice to group and access sets of 
column-value pairs. In terms of our data model, column families are 
optional. They can be dynamically assigned as the row is created. Let a 
base table row $a \in A$ be defined as $a=(k,\{\langle 
c_1,v_1\rangle..\langle c_n,v_n\rangle\})$. In this notation, the row 
key $k$ comes first, followed by a set of column-value pairs $\{\langle 
c_1,v_1\rangle..\langle c_n,v_n\rangle\}$ belonging to 
the column family; this more closely resembles a database row and is 
used throughout the remainder of this paper. When using multiple column 
families, we define the table as $A = (K, F_1,...F_n)$. Then the 
assignment of a column-value set to a column family $F_x$ is denoted by 
$\{..\}_x$. The corresponding row would be defined as $a=(k, \{\langle 
c_1,v_1\rangle..\langle c_i, v_i\rangle\}_1..., \{\langle 
c_{i+1},v_{i+1}\rangle.. \langle c_n,v_n\rangle\}_n)$. 

\begin{figure}[h!] 
	\centering 
	\includegraphics[width=\linewidth]{figures/SystemOverview}  
	\caption{System Overview} 
	\label{fig:system_overview} 
\end{figure} 



The KV-store writes operations, that is client requests, to the TL,
but not entire table rows.  In contrast, a table row stores the row
state, which may result from multiple client requests.  Then, an 
operation $t \in T$ can easily be defined over table row $a \in A$,
with $T = type(A)$  and $type \in \{put, delete\}$. A put operation in
the transaction log is denoted as $t=put (k, \{\langle c_1,v_1\rangle..\langle
c_n,v_n\rangle\})$. A put inserts or updates the column values at the
specified row key. A delete operation $t \in T$ is defined as
$t=delete (k, \{\langle c_1,\emptyset\rangle..\langle
c_n,\emptyset\rangle\})$.  Note that we are leaving the values 
empty; the client just specifies the row key and columns that are to
be deleted. A stream -- respectively, the output of one node's transaction log --
is denoted as a sequence of operations $ts \in TS=(T_1,..,T_n)$. Finally, we can
define the complete output of the KV-store as a set of operation 
streams as $ts_1,..ts_n \in TS$. 

\subsection{View Maintenance System}

%The streams consist of insert and put operation $t_p$ and a delete 
%operation $t_d$ into the base table. These 
%two operation types are the only way a client can modify the base table 
%and, hence, affect the view table. We denote the two operation types, 
%as they are written to the TL, in the general form $t_p=put(A(k,\{\langle 
%c_1, v_1\rangle, \langle c_2,v_2\rangle\}))$ and $t_d=delete( A(k))$. 

The View Maintenance System receives updates from the KV-store in form
of operation streams (see Figure~\ref{fig:system_overview}). Every node
of the KV-store produces a local transaction stream ($ts_1,..,ts_n$);every 
stream of operations is handled by a subsystem of the VMS. The subsystem 
parallizes view update computation: it distributes the updates to a
scalable number of view managers. A view manager actually applies the
update operation to the view table. First, it looks up the view tables 
 defined over the base, then it retrieves the corresponding view record, 
adds the delta to it, and writes back the result to the view.

We express view tables in the VMS with the help of relational algebra 
(e.g., we define a \texttt{SELECTION} as $S=\sigma_{c_1 < x}(A)$ over a 
base table $A$). Defining a view table over a base table is equivalent
to connection the output stream of base table operations with the input
stream of view table input operations. The VMS is also capable of 
defining view tables over view tables (e.g. we define a \texttt{PROJECTION}
view as $P=\pi_{c_1,c_2}(S)$). Thus,
we can concatenate multiple different view types; the VMS will update the
view chain subsequently. It will receive the base table update, apply the
update to the \texttt{SELECTION} view, and apply the result of the first
update to the \texttt{PROJECTION} view.
%
%
% aj - below -
%      We sometimes say view converges, sometimes view consistency
%      We should make sure these concepts are somewhat explained,
%      ideally, before we use them.

\section{View Maintenance System} 
\label{sec:view_maintenance_system} 

In this section, we present and discuss the design of \VMS. By
illustrating how a base table operation may effect a view table, we
provide the intuition for the resulting view consistency established
by \VMS.  Finally, we discuss fault-tolerance.

\subsection{Design Overview}

Figure~\ref{fig:view_maintenance_system} gives an overview of \VMS,
which is comprised of a \textit{coordinator} and an arbitrary number
of \textit{sub-systems}. The coordinator manages system load and
recovery, whereas the sub-systems -- more specifically the
\textit{view managers} (\VMs) in a sub-systems -- update views. The
input to \VMS\ is a set of operation streams ($ts_1$,$ts_2$..$ts_n$);
each emitted by a \KVS\ node (cf. Figure~\ref{fig:kv_model}).  Each
\VMS\ sub-system manages one stream of operations.  A sub-system
distributes the incoming operation stream to \VMs. The number of
\VMs\ per sub-system is configurable. \VMs\ can be dynamically
\textit{assigned} to or \textit{removed} from sub-systems. The
coordinator can also re-assign \VMs\ from one sub-system to another.

A \VM\ is designed to be light-weight and can be deployed in large
numbers to accommodate a changing view update load. It computes view
updates, based on base table operations it receives as input via the
\KVS\ API, for view tables. A \VM\ only belongs to a single
sub-system. The sub-system feeds the \VM\ with operations which
processes them in order.  A \VM\ is the unit of scalability of
\VMS. \VMs\ are kept stateless to be exchangeable at any time and to
minimize dependency. Given a number of view definitions and a sequence
of operations, a \VM\ is always able to execute any view table update
from any host.

\begin{figure}
  \centering
    \includegraphics[width=\linewidth]{figures/ViewMaintenanceSystem}
    \caption{View Maintenance System (\VMS)}
    \label{fig:view_maintenance_system}
\end{figure}

Our design exhibits at least the following four benefits: (i) Seamless
scalability: Hundreds of views may have to be updated as a consequence
of a single base table operation. As \VMS\ exceeds its service levels,
additional \VMs\ can be spawned (below, we show this
experimentally). (ii) Operational flexibility: \VMs\ introduce
flexibility to the system architecture. All \VMs\ of a given
sub-system can be hosted together on the same node or each \VM\ can be
hosted at a different node. (iii) Accommodate load variations:
\VMs\ can be reassigned from one sub-system to another as base table
update load changes. (iv) Fault-tolerance: If a \VM\ crashes, another
\VM\ can take over and continue processing the operation stream.

\subsection{Update Propagation} 
\label{subsec:update_processing} 

A sub-system distributes the arriving base table update stream to its
\VMs\ via consistent hashing by maintaining a hash-ring
(cf. Figure~\ref{fig:review_consistency}), where active \VMs\ are
registered.  Row keys of arriving updates are hashed into the ring and
associated in clock-wise direction with active \VMs.  In this way, a
sub-system distributes operations uniformly across the available
\VMs\ and ensures that base table operations on the same row key are
always handled by the same \VM. On the one hand, this mechanism
ensures maximal degree of concurrency for update propagation, while
simultaneously guaranteeing the ordered propagation of base table
updates to view tables, setting the basis for view table consistency.

After a \VM\ has been selected (via the hash-ring) to process a
certain client operation, the sub-system inserts the operation into
its queue.  One queue for every active \VM\ exists to buffer the
incoming client operations.  A sending thread (which can be
deactivated on demand) fetches the operations one after the other and
sends them to the \VM.

Every \VM\ maintains its own transaction log, referred to as
\textit{\VM-log}. When receiving an operation, a \VM\ directly writes
it to the \VM-log. Just like the transaction log, the \VM-log is kept
available by the underlying file system, employing recovery mechanisms
in face of \VM\ crashes (e.g., in the case of \HB, the file system
redundantly replicates file blocks via HDFS.)

To access and update view tables, a \VM\ acts as a client to the \KVS,
using its standard client API. Given a base table operation (e.g., a
put on a base table $A$), the \VM\ retrieves and caches the view
definitions of the derived views (e.g., a \texttt{SELECTION} and
\texttt{COUNT} view $S$ and $C$, both derived from $A$). Then,
\VM\ runs the update program, and submits view table updates (to $S$
and $C$) via the client API. For some of the view types maintained,
the \VM\ has to query the view table first, as part of the update
logic; in a \texttt{COUNT} view, the \VM\ reads the current count from
the view before applying the delta of the base table operation. These
view queries are always get operations (i.e., single row accesses) and
can be evaluated quickly.


%To access and update a view table, a \VM\ acts as a client to the
%\KVS, using its standard client API. Given a base table operation, the
%\VM\ retrieves (and caches) the view definitions of the views derived
%from that particular base table. Then, the \VM\ runs the update
%program and submits view table updates via the client API. Depending
%on the view type maintained, the \VM\ first queries the view table as
%part of the update logic. For example, in a \texttt{SUM} view, the
%\VM\ reads the current sum from the view before applying the delta of
%the base table operation.  These view queries are always get
%operations (i.e., single row accesses) and are evaluated efficiently
%by the underlying store.

To allow for parallel updates of multiple \VMs\ on the same view
record, we use a test-and-set mechanism (as suggested in
\cite{jacobsen:viewmaintenance}).~\footnote{In \HB\ a
  \texttt{checkAndPut} method is provided to realize this
  mechanism. Most \KVSs\ offer a similar abstraction.} When updating a
table record, a \VM\ sees (tests) whether a record has been
concurrently modified between a read and an update.

\textit{Example:} Let two \VMs, $VM_1$ and $VM_2$ retrieve the value
of the same view record $(x1, \{(col, a)\})$. Let $VM_1$ add the delta
$b$ such that $(x1,\{(col,a+b)\})$ and $VM_2$ add the delta $c$ such
that $(x1,\{(col,a+c)\})$. Now let both \VMs\ write back their values
to the view. As a result one of the delta values (either $b$ or $c$)
is overwritten and is not reflected in the view. Using a test-and-set
method prevents the scenario. Assuming $VM_2$ writes second; when
trying to put the new value, the test-and-set method tests the old
value $a$.  It fails because the old record value changed concurrently
to $a+b$. Thus, $VM_2$ fetches the updated value and re-computes
$(x1,\{(col,a+b+c)\})$. This time, test-and-set succeeds and the
record is written.

%
% aj - I will have to come back to the below paragraph, time-permititng
%
% aj - below -
%      What if a VM crashes after the operation has been written to the VM log
%        and before the view has been updated (state persisted in ZK?)
%      Also, doing the below after every view table update, sounds a bit
%       heavy weight; is this really done?
%      Also, this may not be the place to discuss FT.
%
% ja - the state (of the current committed transaction) has to be persisted
%     in ZK. This definitly draws off some performance, but reliability
%     always comes with a cost. Acutally, i think it's not too bad since 
%     ZK is distributed. Likewise, writing the VM log to HDFS was not a big
%     factor during evaluation.  
%
%	  The only alternative is replaying the log from the region server. 
%	  Still we need to keep track of the current committed transaction in ZK.
%     This is what we favored in the beginning, but it is only convinient
%     with a small number of VMs per region server. Moreover, it
%     causes re-propagation of updates that had been already applied. It
%     changes the read-state of the TL. This is --and i can just speak
%     from intuition here-- error-prone. What has been read from TL and
%	  and transferred to a VM should remain there. Recovery should just
%	  affect the crashed VM and not the rest of the system.
%
%	  In general, this discussion is too detailed to start it here(it may be
%	  also too detailed for FT). I am a little bit clueless how to proceed.
%
%After a view table update, the \VM\ persists the current state of work
%-- that is, the \textit{sequence number} of the last processed client
%operation and the updated view table name. This state is kept in
%ZooKeeper. The coordinator can restore the state in case of a
%\VM\ crash. A new \VM\ can resume the work from this point onward.

\begin{figure}
  \centering
    \includegraphics[width=\linewidth]{figures/ReviewConsistency}
    \caption{Sub-system setup}
    \label{fig:review_consistency}
    \vspace{-2mm}
\end{figure}



\subsection{Consistency Considerations} 

%
% aj - below - here is probably where we should describe what
%              consistency of views is and what convergence
%              means

%
%      We should discuss consistency here for what you call static
%      context and then later dynamic context.
% ja - in progress
%
% aj - in S4, we also say views are correct, smth. we should define
%      as well
% ja - i added a paragraph at the beginning of "consistency considerations"
%
%al.\cite{jacobsen:viewmaintenance} suggest,

Unconstrained incremental view maintenance and parallel propagation of
base table updates to views may result in the materialization of
inconsistent view data. By constraining how updates propagate,
\VMS\ prevents inconsistencies.

Informally speaking and considering that a base table constitutes a
\textit{sequence of base states} (one each resulting from the sequence
of client updates) and a view table a corresponding \textit{sequence
  of view states}. Then, view consistency can be defined via comparing
these two sequences.  Thus, a view is said to \textit{converge}, if a
correct final view state is reached (i.e., a view state that is
identical to evaluating the view expression over the final base table
state). A view is said to be \textit{consistent} if all intermediate
view states are \textit{correct} and \textit{ordered} according to the
base state sequence. That is, there are no view states that could
ever be derived from any base state (correctness). And, there is a
direct correspondence between both state sequences (i.e., racing base
updates will not lead to re-ordered view states)
(order).~\footnote{This description is intentionally kept
  informal. For a detailed formal analysis, see the technical
  reports~\cite{jacobsen:viewmaintenance, extendedpaper}.}
  
Based on the formal considerations in~\cite{jacobsen:viewmaintenance,
  extendedpaper}, we can proof that \VMS, which preserves timeline
of updates, applies view updates exactly once, and avoids ``backward
queries'' (to a base table), achieves
\textit{consistency}\footnote{What we refer to as \textit{consistency}
  here, is defined as \textit{strong consistency}
  in~\cite{jacobsen:viewmaintenance, extendedpaper}, which provides a
  more fine-grained shading of view table consistency.}.  Thus,
\VMS\ guarantees views converge and are consistent. Essentially, this
is the result of the mechanisms introduced above: consistent hashing
ensures that record timelines are respected in update propagation,
test-and-set and recovery mechanisms guarantee that view updates are
applied exactly once, finally, \VMS\ was designed to completely
dispense with backward queries
(cf. Section~\ref{sec:view_maintenance}).

These guarantees hold under the assumption of a ``static''
\VMS\ configuration: A fixed number of nodes and \VMs.  However, given
a ``dynamic'' context, where \VMS\ assigns and withdraws \VMs\ or
allows \KVS\ to add and remove nodes or move key-ranges, results in
the interference with view consistency, which we address below.


\subsubsection{Management Actions and Consistency} 

We now refine the behaviour of \VMS\ to allow for dynamic state
changes in sub-systems.  For reasons of recovery, load balancing and
scaling, \VMS\ performs the following actions: \textit{add}, makes the
\VM\ resource available to the \VMS; \textit{remove}, takes away a
\VM\ resource; \textit{assign}, makes a \VM\ available to a sub-system
such that it can process the sub-system's client operations;
\textit{withdraw}, takes away a \VM\ from a sub-system (opposite of
assign); and \textit{re-assign}, re-locates a \VM\ from one sub-system
to another. Adding a \VM\ to \VMS\ (or removing it) does not affect
view maintenance. The re-assign action is a combination of withdraw
and assign. Now, we explain how assign and withdraw actions are
performed safely (i.e., without interfering with consistency).

%\begin{table}
%\rowcolors{2}{gray!10}{gray!30}
%\setlength{\belowrulesep}{0pt}
%\setlength{\aboverulesep}{0pt}
%\setlength\extrarowheight{2pt}
%\begin{center}
%\begin{tabular}{l l l}
%\toprule
%Component & action & method \\
%\midrule
%View Manager & add & \textit{addViewManager()}  \\
% & remove & \textit{removeViewManager()}    \\
% & assign & \textit{assignViewManager()}  \\
% & withdraw & \textit{withdrawViewManager()}    \\
% & re-assign & \textit{reassignViewManager()}    \\
%\bottomrule 
%\end{tabular}
%\caption{\VMS\ actions}
%\label{tab:vms_events}
%\end{center}
%\end{table}

%Recall, that the sub-system selects the 
%responsible \VM\ by applying the hash function. The operation is then 
%inserted into the corresponding queue. 


\noindent
\textbf{Assign VM:} Figure~\ref{fig:review_consistency} shows how a
sub-system does view maintenance. Assume a new \VM\ (in the figure
represented by $V_3$) is assigned to the sub-system. The logic,
performed on the sub-system during the command, can be described as a
sequence of primitive actions: (1) Method $createQueue(vm)$ creates a
queue for a new \VM. (2) The \VM\ is added to the hash-ring by method
$insertHash(vm)$.  (3) The method $activateQueue(vm)$ starts the
sending thread that keeps transferring the queue's operations to the
\VM.

When a \VM\ is assigned, it is inserted into the hash-ring of the 
sub-system. Unless the hash-ring is empty, the new \VM\ is assigned a key 
range that is, at the same time, removed from another \VM. This leads
to violation of consistency (even convergence in terms of the consistency
model), as the following example shows.


\begin{algorithm}
\caption{Safe assignment procedure at sub-system}
\label{alg:assignvm}
\begin{algorithmic}
\Procedure{$assignVm$}{$vm_a, VM_{sub}$}
\State{$createQueue(vm_a)$}
\State{$insertHash(vm_a)$}
\ForAll{$vm \in VM_{sub}$}
\State{$queue(vm, m_a)$}\Comment{queue markers}	
\EndFor
\ForAll{$vm \in VM_{sub}$}
\State{$receiveAck(vm)$}\Comment{wait for acks}		
\EndFor
\State{$activateQueue(vm_a)$}	
\EndProcedure
\end{algorithmic}
\end{algorithm}


\textit{Example}: In Figure~\ref{fig:review_consistency}, $VM_1$ and
$VM_2$ are already assigned to the sub-system. They are responsible
for a certain range on the hash-ring; queues are feeding them with
incoming client operations. Assume a client performing a put operation
$p_1(k_1, \{..\})$; the sub-system receives $p_1$, selects $VM_2$ as
responsible and inserts $p_1$ into the queue of $VM_2$.  Now assume, a
new \VM\ $VM_3$ is assigned to the sub-system and its hash is inserted
within the range of $VM_2$ such that it acquires the responsibility
for key $k_1$. In a next step, a client sends an operation $p_2(k_1,
\{..\})$ to the same key. Because responsibility has changed, the
operation is sent to $VM_3$. Considering that $VM_3$ has just been
added, it's queue is empty and, hence, processes updates very fast. It
is likely to happen that $VM_3$ processes $p_2$ before $VM_2$ can
process $p_1$. Because both operations refer to the same base record,
the timeline of the record is broken and convergence is violated.

To process the assign command safely, we use so called \textit{markers}. 
Markers are -- just like client operations -- enqueued to a \VM\ and 
become a part of the operation stream. When the \VM, while processing 
operations, notices a marker, it replies with an acknowledgement back to 
the sub-system. Thus, the marker reveals to the sub-system, whether the \VM\ 
has completed all operations that were sent before the marker. We change the 
assignment procedure by adding a marker-based acknowledgement mechanism 
(cf. Algorithm~\ref{alg:assignvm}). 

%The
%algorithm is executed synchronously and if another assignment
%procedure is called on the sub-system, it must wait, until the first
%operation has terminated.
The procedure $assignVm$ takes two parameters: $vm_a$, the \VM\ that 
should be assigned to the sub-system, and $VM_{sub}$, a set of \VMs\ 
that are already assigned to the sub-system. The algorithm creates a 
queue for $vm_a$ and inserts it into the hash-ring. Then, it queues a 
marker $m_a$ to all assigned \VMs\ ($VM_{sub}$). After the sub-system 
has received acknowledgements from all \VMs, it is guaranteed that no 
operation in the key range of the newly added \VM\ $vm_a$ is still 
pending. Referring back to the above example: The timeline of $k_1$ can 
not be changed any more. 

%Operation $t_2$ has to
%wait in the queue of $VM_3$ until $VM_2$ acknowledges the processing
%of $t_1$ because the queue of $VM_3$ is activated only after the
%marker has been acknowledged and this, in turn, implies that operation
%$t_1$ has been processed.
\noindent
\textbf{Withdraw VM:} The logic, performed on the sub-system side
during a withdraw can be described analogously to the assign command
by a sequence of primitives (opposite actions in reverse order): (1)
$deactivateQueue(vm)$ stops the sending thread that keeps transferring
the operations in the queue to the \VM. (2) The \VM\ is removed from
the hash-ring by $removeHash(vm)$. (3) $deleteQueue(vm)$ removes the
queue for the withdrawn \VM. The queue can only be deleted, if it is
empty and no operation is queued.

\begin{algorithm}
\caption{Safe withdraw procedure at sub-system}
\label{alg:withdrawvm}
\begin{algorithmic}
\Procedure{$withdrawVm$}{$vm_w, VM_{sub}$}
\ForAll{$vm \in VM_{sub} \wedge vm \neq vm_w$}	
\State{$deactivateQueue(vm)$}
\EndFor
\State{$removeHash(vm_w)$}
\State{$queue(vm_w, m_w)$}\Comment{queue marker}
%\ForAll{$vm \in VM_{sub}$}
\State{$receiveAck(vm_w)$}\Comment{wait for ack}		
%\EndFor
\State{$removeQueue(vm_w)$}
\ForAll{$vm \in VM$}
\State{$activateQueue(vm)$}
\EndFor
\EndProcedure
\end{algorithmic}
\end{algorithm}

    
Also, during a withdraw procedure, consistency may be violated. At the
moment where a \VM\ is withdrawn, i.e., removed from the hash-ring,
its queue might still contain operations. If another \VM\ that
acquires the key range is fast enough, it might processes operations
before the withdrawn \VM\ has finished. Again, the timeline of base
records is changed. In order to prevent inconsistencies, we designed
Algorithm~\ref{alg:withdrawvm} analogously to
Algorithm~\ref{alg:assignvm}. It takes two parameters: $vm_w$, the
\VM\ that should be withdrawn from the sub-system, and $VM_{sub}$, the
set of \VMs\ that is assigned to the sub-system.

First, the queues of the \VMs\ that possibly increase their key range
on the hash-ring (i.e., all other \VMs) are deactivated. Then, a
marker $m_w$ is queued at the \VM\ that is withdrawn. If the \VM\ has
acknowledged the marker, the sub-system knows, that all operations
have been processed. It removes the queue of the \VM\ and re-activates
the queues of all \VMs.

\subsubsection{Fault-tolerance} 

Failure detection and recovery play a critical role in \VMS.  In this
section, we analyse the behaviour of \VMS\ under \VM\ and node crashes
to ensure that after appropriate recovery measures, views still
converge.

\noindent
\textbf{\VM\ crash} -- A \VM\ maintains a queue with operations
dispatched to it. During a \VM\ crash, these operations are lost, which
may result in non-converging views. Our recovery measures described
here, ensure view convergence under \VM\ crash. A \VM\ crash triggers an
event via ZooKeeper, notifying the \VMS\ coordinator.

First, the coordinator sends a withdraw command to the concerned
sub-system. The sub-system withdraws the crashed \VM\ from the
hash-ring and stops dispatching operations to it. This way no updates,
that were in-fight, while the \VM\ crashed, are lost. Next, the
coordinator starts a new \VM\ instance. Upon start-up, it tells the
new \VM\ to replay the \VM-log of the crashed \VM. The new
\VM\ contacts ZooKeeper and retrieves the last processed sequence
number of the crashed
\VM\ (cf. Section~\ref{subsec:update_processing}).  The new
\VM\ accesses the \VM\ log of the crashed \VM\ and -- starting from
the sequence number -- replays all the entries.

\noindent
\textbf{Node crash} -- A node crash is handled by the recovery
mechanism of the \KVS. The \KVS\ moves all key ranges of the crashed
node to other nodes. In case, client operations exist that were just
residing in memstore (and had not been written to the table file), the
\KVS\ replays the\TL. During replay, all operations are inserted into
memstore and directly flushed to disk.

The \TL\ of a crashed node is still available (due to replication in
the underlying file system, HDFS, for example.) Thus, the sub-system
that is streaming the operations from the crashed node's
\TL\ continues reading to the end of file. Now, that the stream (of
the crashed node's \TL) runs dry, \VMS\ re-assigns all \VMs\ to a
different sub-system.

Based on the above reasoning, we conclude that \VMS\ is able to
prevent loss and duplication of operations during crashes.



%
%\input{sections/consistency}
%
\input{sections/view_maintenance_concept}

%\subsection{Proof of Consistency}


Theorem~\ref{theo:strong_consistency} states that a VMS system fulfilling all 
three requirements achieves at least strong consistency. In the following, we 
will present a proof for this theorem, which is organized in three stages: we 
start with proving convergence and then present extensions to also prove weak 
consistency  and finally strong consistency.





\subsubsection{Notation}
First, we define the following notation for keys, 
operations on keys, and the ordering of operations. Let $k_x$ denote a key in 
a base table, where $x \in X = \{1, \dots, m\}$, and $X$ is the 
table's key range. Further, let an operation on key $k_x$ be defined as 
$t[k_x]$, and a totally ordered sequence of such operations be denoted by 
$\langle t_1, t_2, t_{3}, \dots, t_N\rangle$, where $N$ defines the total 
number of operations on the table in a given timespan. 
Hence, a generalized sequence of operations on a base table is 
represented by $\langle t_1[k_{x_1}], t_2[k_{x_2}], \dots, 
t_n[k_{x_m}]\rangle$, where $k_{x_1}, \dots, k_{x_m} \in X$ can be arbitrarily 
chosen from the base table's key range for every timestamp. Based on this 
generalized sequence every other sequence of operations can be derived, e.g. 
$\langle t_1[k_{1}],t_2[k_{2}],t_3[k_{1}]\rangle$. In some cases, we also want 
to keep the ordering of operations variable. For that reason, we introduce an 
index $s_i$ with $i \in \{1, \dots, n\}$. Then we can write the arbitrary 
sequence as $\langle t_{s_1}[k_{x_1}], t_{s_2}[k_{x_2}], \dots, 
t_{s_n}[k_{x_n}]\rangle$ with $s_1\neq \dots \neq s_n$. Using this notion, we 
are capable of representing every possible sequence of update operations in 
the system. 

The index $t^{(i)}[k_x]$ is used to express a sequence of operations on 
a single row key (i.e. the time-line). For example, a sequence of 
operations on row key $k_x$ is denoted as $\langle 
t^{(1)}[k_x],t^{(2)}[k_x]...,t^{( \omega)}[k_x]\rangle$. The last 
operation on a particular row key is always denoted with $\omega$. (see 
Section~\ref{sec:consistency}).

For the proof, we assume such an arbitrary sequence of base table operations
 and then we show that --- given the requirements in the theorem --- a VMS 
 system will produce correct view results. Formally, we show that $V_f=View(B_f)$. 

\subsubsection{Convergence}
\label{sub:proof_convergence}
As mentioned earlier, every view type defines its own mapping from base table 
to view table records. Thus, we prove the different cases separately. 

\noindent
\textit{One-to-one mapping:} \texttt{SELECTION}, \texttt{PROJECTION} 
views define a one-to-one mapping between base and view table. The row 
key of the base table is also the row key of the view table. Operations 
for both view types are idempotent, meaning an operation could be 
applied several times without changing the result. The view record is 
always a representation of the last base table operation applied. A 
correct view record with row key $k$ is defined as the last operation on 
the row key in the base table key, e.g. $k\leftarrow 
View(t^{(\omega)}[k])$. The function $View$ calculates the view record 
(by using the appropriate view definition) and applies the result to the 
correct view key. A view table converges, if all view records are 
computed correctly in the last view state. 

We start defining an arbitrary sequence of operations on the base table, 
shown in Step~\ref{proof:oo_step1}. Clients can update different row 
keys in the base table, using put or delete operations. Likewise, they 
can update the same row key multiple times. The update operations of the 
clients form a particular global order, expressed through $t_1..t_n$. 
They take the base table from its initial state $B_0$ to its final state 
$B_f$. In the next step, all update operations get forwarded, causing 
the ordering of operations to be lost. If we would continue working with 
an unordered set, convergence of the view will be violated at some 
point. For this reason, we apply requirement (iii) (,i.e. the time-line 
requirement) to our equation, as depicted in Step~\ref{proof:oo_step3}. 
As updates operations do not influence each other (see requirement 
(ii)), we are able to create a set of sub-sequences. These 
sub-sequences only consist of updates that have been applied to the same 
row key. Likewise the sub-sequences contain all operations from the 
previous step (see requirement (i)). 
%
\begin{subequations}
  \begin{align}
 S_b&=\langle t_1[k_{x_1}],....,t_n[k_{x_n}] \rangle;\;\;\Big(B_0 \overset{S_b}{\rightarrow}B_f\Big) \label{proof:oo_step1}\\ 
 S_1&=\langle t^{(1)}[k_{1}],..,t^{(\omega_1)}[k_{1}]\rangle,..,S_m=\langle t^{(1)}[k_{m}],..,t^{(\omega_m)}[k_{m}]\rangle\label{proof:oo_step3}\\
  S_1&=\langle t^{(\omega_1)}[k_{1}]\rangle,..,S_m=\langle t^{(\omega_m)}[k_{m}]\rangle\label{proof:oo_step4}\\
 S_v&=\langle t_{s_1}^{(\omega_1)}[k_{1}],..t_{s_n}^{(\omega_m)}[k_{m}]\rangle;\; \Big(V_0\overset{S_v}{\rightarrow}V_f\Big)\label{proof:oo_step5}\\
 	V_f&=k_x\leftarrow View(t^{(\omega_x)}[k_x])=View(B_f)\label{proof:oo_step6}
  \end{align}
\end{subequations}
%
In Step~\ref{proof:oo_step4}, we pick the last element of all 
sub-sequences and eliminate the rest. As stated above, only the last 
operation on a particular row key has an influence on the final view 
state. Again, we observe that the time-line of a row key is vital. If it 
is broken, e.g. for row key $k_1$, then an operation 
$t^{(\omega-1)}[k_{1}]$ can be incorporated into the final result and 
render it incorrect. After the elimination, we unite the operations 
again in Step~\ref{proof:oo_step5}. The reunion allows the remaining 
operations to be executed in every possible order (i.e. every operation 
could be executed in parallel). Finally, the view definition is applied 
to every operation in $R$. This leads to the correct final view records 
and hence, to convergence. The \texttt{DELTA} view also defines a 
one-to-one mapping between base and view records. In contrast to the 
aforementioned views, the results of the \texttt{DELTA} view do not only 
relate to the last, but to the two last operations. Therefore, we need 
to change the last three steps of the proof as follows: 
%
\begin{subequations}
  \begin{align}
  S_1&=\langle t^{(\omega_1-1)}[k_{1}],t^{(\omega_1)}[k_{1}]\rangle,..,S_m=\langle t^{(\omega_m-1)}[k_{m}],t^{(\omega_m)}[k_{m}]\rangle\\
   S_v&=\langle t_{s_1}^{(\omega_1-1)}[k_{1}],t_{s_2}^{(\omega_1)}[k_{1}],..t_{s_{n-1}}^{(\omega_m-1)}[k_{m}],t_{s_n}^{(\omega_m)}[k_{m}]\rangle\label{proof:ood_step2}\\
   &(\forall t_{s_i}^{(\omega_y-1)}\in S_v)(\exists t_{s_j}^{(\omega_y)}\in S_v)\;s_i < s_j;\;\Big(V_0\overset{S_v}{\rightarrow}V_f\Big)\notag\\
 	V_f&=k_x\leftarrow View(t^{(\omega_x-1)}[k_x],t^{(\omega_x)}[k_x])=View(B_f)
  \end{align}
\end{subequations}
%
As can be observed in Step~\ref{proof:ood_step2}, the two last 
operations of a time-line are included into the final result. However, 
the sequence can be arbitrarily ordered, but needs to preserve the 
time-line of both last operations (i.e. $\omega-1$ must always precede 
$\omega$). Computing $V_f$ leads to the valid final state and the 
\texttt{DELTA} view converges. 

\noindent
\textit{Many-to-one mapping:} (\texttt{PRE}-)\texttt{AGGREGATION} and 
\texttt{INDEX} views define a many-to-one mapping between base and view 
table. The row key of the view table is the aggregation key. Multiple 
row keys in the base table can relate to a particular aggregation key. 
However, a base table row has always only one aggregation key. A correct 
view record with aggregation key $x$ is defined as the combination of 
multiple base records $k_{x_1}..k_{x_j}$, related to the particular key. 
In terms of incremental view maintenance, the correct view record can be 
defined as a number of last operations, that have been applied to this 
combination of base records: $x \leftarrow 
View(t^{(\omega_1)}[k_{x_1}],..,t^{(\omega_j)}[k_{x_j}])$. In case of a 
\texttt{SUM} view e.g., this resolves to $x \leftarrow 
f(t^{(\omega_1)}[k_1])+..+f(t^{(\omega_j)}[k_j])$. We start again, 
defining an arbitrary sequence of base table operations in 
Step~\ref{proof:mo_step1}. In contrast to the previously handled views, 
we are now processing $\delta$-operations. We construct a number of $m$ 
sub- sequences, each containing the $\delta$-operations of one 
particular base record key. In Step~\ref{proof:mo_step2}, we merge the 
$\delta$-operations together. All $\delta$- operations add up to form 
the last transaction as the end result (i.e. $\delta(t^{(1)}[k_{1}])+
..+\delta(t^{( \omega_1)}[k_{1}])=t^{(\omega_1)}[k_{1}]$). 
%\begin{subequations}
%  \begin{align}
%  \{t_1(k_1),..,t_n(k_1)..t_1(k_n),..,t_n(k_n)\}\\
% \{\langle t_1(k_1),..,t_n(k_1)\rangle,..\langle t_1(k_n),..,t_n(k_n)\rangle\}\\
% \{\langle t_n(k_1)\rangle,..\langle t_n(k_n)\rangle\}\\
% 	x=f(t_n(k_1))+..+f(t_n(k_n))
%  \end{align}
%\end{subequations}
%
\begin{subequations}
  \begin{align}
  S_b&=\langle t_1[k_{x_1}],....,t_n[k_{x_n}] \rangle;\;\Big(B_0 \overset{S_b}{\rightarrow}B_f\Big) \label{proof:mo_step1}\\ 
 S_1&=\langle \delta(t^{(1)}[k_{1}]),.,\delta(t^{(\omega_1)}[k_{1}])\rangle,..,\label{proof:mo_step2}\\
 &\hspace{10 mm}S_m=\langle \delta(t^{(1)}[k_{m}]),.,\delta(t^{(\omega_m)}[k_{m}])\rangle\notag\\
  S_1&=\langle t^{(\omega_1)}[k_{1}]\rangle,..,S_m=\langle t^{(\omega_m)}[k_{m}]\rangle\\
 S_v&=\langle t_{s_1}^{(\omega_1)}[k_{1}],..t_{s_n}^{(\omega_m)}[k_{m}]\rangle;\;\Big(V_0\overset{S_v}{\rightarrow}V_f\Big)\label{proof:mo_step4}\\
 	V_f&=x\leftarrow View(t^{(\omega_{x_1})}[k_{x_1}],.., t^{(\omega_{x_j})}[k_{x_j}])=View(B_f)\label{proof:mo_step5}
 	%&\hspace{10 mm}x_1,..,x_j \in \{1,..,m\};\;x_1\neq,..,\neq x_j \notag
  \end{align}
\end{subequations}
%
Now, we can unite the single sequences as done before. We retrieve a 
final sequence as shown in Step~\ref{proof:mo_step4}. The operations of 
this sequence are then applied to the view --- simultaneously they are 
grouped and stored according to their aggregation key. The final view 
records are calculated correctly, as depicted in 
Step~\ref{proof:mo_step5}, which causes the aggregation view to 
converge. 

\textit{Many-to-many mapping:} (\texttt{REVERSE}-)\texttt{JOIN} views 
define a many-to-many mapping between base and view table. The row key 
of the view table is a composite key of both join tables' row key. 
Multiple records of both base tables form a set of multiple view records 
in the view table. Since the joining of tables takes place in the 
\texttt{REVERSE JOIN} view, we prove convergence only for this view 
type. A \texttt{REVERSE JOIN} view has a structure that is similar to an 
aggregation view. The row key of the \texttt{REVERSE JOIN} view is the 
join key of both tables. All base table records are grouped according to 
this join key. But in contrast to an aggregation view the 
\texttt{REVERSE JOIN} view combines two base tables to create one view 
table. A correct view record with join key $x$ is defined as a 
combination of operations on keys $k_1..k_n$ from join table $A$ and 
operations on keys $l_1..l_p$ from join table $B$. In order to represent 
both keys we introduce an additional variable $z_1,..,z_n \in 
\{k_1,..,k_m, l_1,..,l_p\}$. Then, the correct view record is defined 
as: $x \leftarrow View(t^{(\omega_1)}[z_1], ..,t^{(\omega_j)}[z_j])$. 
We start with a sequence of arbitrary client updates to both base 
tables, as depicted in Step~\ref{proof:mm_step1}. Then, the order of 
updates is lost and the time-line requirement is realized in 
Step~\ref{proof:mm_step2}. 
%
\begin{subequations}
  \begin{align}
  S_b&=\langle t_1[z_1],..,t_n[z_n]\rangle;\;\Big(B_0 \overset{S_b}{\rightarrow}B_f\Big)\label{proof:mm_step1}\\ 
 S_1&=\langle \delta(t^{(1)}[k_{1}]),..,\delta(t^{(\omega_1)}[k_{1}])\rangle,..,S_m,..,S_{m+1},..\label{proof:mm_step2}\\ 
&\hspace{10 mm}S_{m+p}=\langle \delta(t^{(1)}[l_{p}]),..,\delta(t^{(\omega_p)}[l_{p}])\rangle\notag\\
  S_1&=\langle t^{(\omega_{k_1})}[k_{1}]\rangle,.., S_m=\langle t^{(\omega_{k_m})}[k_{m}]\rangle,..,\\
 &\hspace{10 mm}S_{m+1}=\langle t^{(\omega_{l_1})}[l_{1}]\rangle,..,S_{m+p}=\langle t^{(\omega_{l_p})}[l_{p}]\rangle\notag\\
 S_v&=\langle t^{(\omega_{k_1})}[k_{1}],..t^{(\omega_{k_m})}[k_{m}],\\
 &\hspace{10 mm}t^{(\omega_{l_1})}[l_{1}],..,t^{(\omega_{l_p})}[l_{p}] \rangle;\;\Big(V_0\overset{S_v}{\rightarrow}V_f\Big)\label{proof:mm_step3}\\
 	V_f&=x\leftarrow View(t^{(\omega_{z_1})}[z_1],.., t^{(\omega_{z_j})}[z_j])=View(B_f)\label{proof:mm_step4}
 	%&\hspace{10 mm}z_1,..,z_j \in \{k_1,..,k_m,l_1,..,l_p\}; z_1\neq,..,\neq z_j ;\; \notag
  \end{align}
\end{subequations}
%
We eliminate all but the last operations $\omega$ and reunite the 
operations in Step~\ref{proof:mm_step3}. This leads to the final 
Step~\ref{proof:mm_step4}, where the operations are applied to the view 
record. Since the view records are calculated correctly (i.e. only the 
last operations of the row keys are included) we conclude that the view 
converges. 

\subsubsection{Weak consistency} 
\label{sub:proof_weak}

Weak consistency has been defined as follows: Weak consistency is given 
if the view converges and all intermediary view states are valid, 
meaning they can be derived from one of the base states with 
$V_j=View(B_i)$. As we already proved convergence, we need show that all 
the intermediary view states are correct likewise. We start again with 
an arbitrary sequence of operations in Step~\ref{proofw:oo_step1}. In 
order to generate an intermediate base state, we cut the sequence at any 
point before an operation $t_a$, with $1 < a < n$. After the ordering is 
lost, we apply the time-line consistency. But in contrast to before, we 
are not capable of processing the complete time-line (i.e. 
$(1)..(\omega)$). Instead, we process the time-line until an 
intermediary element $\alpha_x \leq \omega_x$. 
%
\begin{subequations}
  \begin{align}
  S_b&=\langle t_1[k_{x_1}],....,t_n[k_{x_n}] \rangle;\;\Big(B_0 \overset{S_b}{\rightarrow}B_{a}\Big)\label{proofw:oo_step1}\\
 S_1&=\langle t^{(1)}[k_{1}],..,t^{(\alpha_1)}[k_{1}]\rangle,..,S_m=\langle t^{(1)}[k_{m}],..,t^{(\alpha_m)}[k_{m}]\rangle\\
 S_v&=\langle t_{s_1}^{(1)}[k_{1}],..t_{s_i}^{(\alpha_1)}[k_{1}],..,t_{s_j}^{(1)}[k_{m}],..,t_{s_a}^{(\alpha_m)}[k_{m}]\rangle;\;\Big(V_0\overset{S_v}{\rightarrow}V_a\Big)\\
 (\forall& t_{s_1}^{(\lambda_1)}[k_{x_1}] \in S_v)(\forall t_{s_2}^{(\lambda_2)}[k_{x_2}] \in S_v):(x_1=x_2)\;\land\;(\lambda_1<\lambda_2) \Rightarrow s_1 < s_2\notag\\
V_{a}&=k_x\leftarrow View(t^{(\alpha_x)}[k_x])=View(B_{\alpha})
  \end{align}
\end{subequations}


\subsubsection{Strong consistency}
\label{sub:proof_strong}

Strong consistency has been defined as follows: Weak consistency is 
achieved and the following conditions hold true. All pairs of view 
states $V_i$ and $V_j$ that are in a relation $V_i \leq V_j$ are derived 
from base states $Bi$ and $B_j$ that are also in a relation $B_i \leq 
B_j$. Since weak consistency is already proven, we only need to prove 
the statement $V_i \leq V_j \Rightarrow B_i \leq B_j$. If this statement 
is negated, then only two of the following cases can occur: Either $V_i 
\leq V_j \Rightarrow B_i \parallel B_j$ or $V_i \leq V_j \Rightarrow B_i 
\geq B_j$. Both cases can only be constructed by breaking the record 
time-line. To be precise: At least one record has to exists, whose 
time-line is broken. Formally, we demand $ (\exists t_l \in B_i)(\forall 
t_k \in B_j):(r(t_l)=r(t_k))\land(l > k) $. Because requirement (iv) 
prevents the breaking of time-lines, we conclude that both cases are not 
possible. Thus, we have proven strong consistency by contradiction. 




%\section{Fault Tolerance}
\label{sec:fault_tolerance}

Failure detection and recovery play a critical role, especially in
large-scale distributed systems.  In this section, we analyze the
behavior of VMS under VM and node crashes to ensure that after
appropriate recovery measures, views still converge.

\noindent
\textbf{VM crash} -- A VM maintains a queue with operations dispatched
to it. Upon a VM crash, these updates are lost, which may result in
non-converging views. Our recovery measures described here, ensure
view convergence under VM crash. For every update a VM propagates, it
stores the sequence ID of the processed operation in its commit log. A
VM crash triggers an event via Zookeeper
(cf. Section~\ref{sec:system_overview}), notifying the VMS
coordinator. Via the commit log, the coordinator retrieves the last
operation the crashed VM successfully propagated. The commit log order
is determined by the order the VM propagates updates, which results
from the stream of operations dispatched to it, respecting record
timeline semantics. Now, the coordinator executes the steps of
Algorithm~\ref{alg:crashed_vm}.



\begin{algorithm}
\caption{VM recovery in VMS}
\label{alg:crashed_vm}
\begin{algorithmic}[5]
\Procedure{$onVMCrashed$}{$nx, VM_{nx}, vm_c$}
\State $sendMessage(nx, withdraw, vm_c)$
\ForAll{$vm \in VM_{nx}$}	
\State $sendMessage(vm, requestSID)$
\EndFor
\ForAll{$vm \in VM_{nx}$}	
\State $SIDs \leftarrow receiveMessage(vm, SID)$	
\EndFor
\State $smallestSID \leftarrow minimum(SIDs)$
\State $sendMessage(nx, replay, smallestSeqID)$	
\EndProcedure
\end{algorithmic}
\end{algorithm}

First, the coordinator sends a withdraw command to the 
concerned NX component. The NX withdraws the crashed VM from the 
hash-ring and stops dispatching operations to it. This way no updates, 
that were in-fight while the VM crashed, are lost. Next, the coordinator 
obtains the last sequence ID of every VM (assigned to the node the 
crashed VM belonged to) and determines the smallest one, including the 
one from the crashed VM's commit log. The VMS advises the NX component 
to replay the log entries from the point of the smallest ID. This 
recovery measure is necessary for the following reasons: It might happen 
that a running view manager is executing a base table operation with a 
smaller sequence ID than the crashed one. Then, a replay from the crashed 
view manager's ID would cut off some of the client operations. Taking 
the smallest sequence ID, the VMS ensures that no operation gets lost.
However, there might be a case when the crashed view manager had already 
queued up operations but not written anything to the commit log.  Then 
the recovery of the sequence ID fails and again, operations may be lost.
The VM can just avoid this case by writing the first sequence ID it ever
receives from a node, directly to the commit log, marking it as 'not 
processed'. Then the transaction log can be replayed from this point. 
Another problem during replay is that some operations may be forwarded 
to the view managers again. This leads to duplicate view updates. 
But thanks to the signature mechanism (cf.Section~\ref{sec:consistency}), 
the duplicate operations can be identified and just ignored. 


\noindent
\textbf{Node crash} -- A node crash is handled by the recovery
mechanism of the KV-store. It selects a new node and replay's the
crashed node's TL. The coordinator of VMS just notices the
crash  and re-assigns the idle view managers to another
node.



\section{Evaluation}
\label{sec:evaluation}

In this section, we report on the results of an extensive experimental
evaluation of our approach.  We fully implemented VMS in Java and
integrated it with Apache HBase. Before we discuss our experiments, we
briefly review the experimental set-up and the generated workload.

%
% aj - this section - if we can afford it space-wise, it would be better to
%                              use 3 subsections for Impl., Setup, and Experiments
%\subsection{Implementation} 
%The integration of VMS and HBase involves
%two simple steps.  First, we integrate the NX component in the HBase
%architecture. Second, we let the VM component interact with HBase via
%its client API. Figure~\ref{fig:system_overview} shows the
%integration.
%
%In HBase parlance a \textit{region server} (i.e., a node) writes
%client updates to a TL (HBase calls it write-ahead log), which the NX
%simply reads via the file system API, as described in
%Section~\ref{sec:kv_model}.  HBase uses the HDFS file system, where TL
%and table files are persisted for every node.  We use the HBase
%internal class \texttt{HLog.Reader} to interpret the log content and
%retrieve entries in a readable form, as they are serialized and not
%stored in plain text.  NX extracts the row key and routes the
%operation to the appropriate VM. A single TL entry in HBase contains
%exactly the information VMS expects, i.e., row key, table name,
%etc. The difficulties of not being able to differentiate inserts from
%updates and not having access to old column values via the log are the
%as in our model.  An instance of NX is deployed with every node.
%
%As a VM starts processing operations and materializing views, it
%connects to the HBase API to submit view table updates as prescribed
%by our design.  A VM uses an abstract class \texttt{KVSTableService},
%which contains all methods required to access the KV-store and to
%materialize each view type. For each view type, these methods are:
%\texttt{put()}, \texttt{get()}, \texttt{delete()},
%\texttt{checkAndPut()} and \texttt{checkAndDelete()}. Now, the VM is
%able to access HBase and update view tables.
%
%As can be seen from this discussion VMS and the underlying KV-store
%are neatly decoupled, requiring only two integration points (client
%API and TL).

%
% aj - below - can we justify the setup a bit; why that distribution of nodes?
%                   - did we try out other distributions and this was the best?
% This is minor.
%

\subsection{Experimental set-up}  
All experiments were performed on a
cluster comprised of 40 nodes (running Ubuntu 14.04). Out of these, 11
machines were dedicated to the Apache Hadoop (v1.2.1) installation,
one machine as name node (HDFS master) and 10 machines as data nodes
(HDFS file system). On HDFS, we installed HBase (v0.98.6.1) with one
master and 10 region servers. On every region server, we deployed one 
of our extensions (cf. Figure~\ref{fig:kv_model}). View managers (VMs) 
were deployed on 20 separate
machines to be able to scale them without interfering with the core
system (i.e., the 12 HBase nodes.) Also, running multiple VMs on the
same machine is possible, since, as we observed, the limiting factor
for a VM is the access latency to HBase (i.e., request processing
latency).  Finally, another 8 nodes were reserved for HBase clients to
simulate the load on the base tables by constantly issuing updates.

\subsection{Workload} 
Prior to each experiment, we created an empty
base table and a set of view tables. Concrete view definitions as well
as the assignment of view tables to base tables are maintained as meta
data in a separate \textit{view definition} table. By default, HBase
stores all base table records in one region. We configured
HBase to split every table into 50 parts -- which lets HBase balance 
the regions with high granularity and ensures an uniform
distribution of keys among available region servers. 

\begin{figure*}
\minipage{0.32\textwidth}
  \includegraphics[width=\linewidth]{figures/selection}
      \vspace{-5mm}
  \caption{Selection performance}\label{fig:selection}
        \vspace{-5mm}
\endminipage\hfill
\minipage{0.32\textwidth}
  \includegraphics[width=\linewidth]{figures/aggregations}
      \vspace{-5mm}
  \caption{Aggregation performance}\label{fig:viewtypes}
        \vspace{-5mm}
\endminipage\hfill
\minipage{0.32\textwidth}%
  \includegraphics[width=\linewidth]{figures/join}
      \vspace{-5mm}
  \caption{Join performance}\label{fig:join}
        \vspace{-5mm}
\endminipage
\end{figure*}

For \texttt{COUNT}, \texttt{SUM}, \texttt{MIN}, and \texttt{MAX}
views, we create base tables that contain one column $c_1$
(aggregation key) and another column $c_2$ (aggregation value).  We
choose a random number $r_a$ between $1$ and some upper bound $U$ to
generate aggregation keys.  Thereby, it is possible to vary the number
of base table records that affect one particular view table
record. For the \texttt{SELECTION} view, we use the same base table
layout and apply the selection condition to column $c_2$. The
\texttt{JOIN} view requires two base tables with different row
keys. The row key of the right table is stored in a column in the left
table, referred to as foreign key.

Basically, the workload we generate for our experiments consists of
\textit{insert}, \textit{update} and \textit{delete} operations that
are issued to HBase using its client API. Operations are generated
according to different distributions over the key space (we use Zipf
and Uniform).  A Zipf distribution, for instance, simulates a ``hot
data'' scenario, where only few base table records are updated very
frequently.


\subsection{Experiments} 

With our experiments, we primarily evaluate the performance of the
system with regards to throughput and view maintenance latency.

\subsubsection{Impact of view type on throughput} 
First, we evaluate the
performance for every view type, separately. We measure the throughput
during view maintenance, i.e., the number of updates a view can
sustain per second.  We configure a system with a fixed number of 10
region servers to host all tables. The number of VMs varies between 10
to 50, and all VMs are assigned evenly to region servers. Furthermore,
40 clients generate a total of 1 million operations per
experiment. Updates are concurrently processed by all VMs. The
experiments are completed after all clients sent their updates and all
VMs emptied their queues.

For the \texttt{SELECTION} view, we experimented with three different
selectivity values(i.e., selection of 10, 50 and 90 percent of base
records), while varying the number of VMs that process the update
load. Figure~\ref{fig:selection} shows the results.  The
\texttt{SELECTION} view is not realized with an auxiliary
view. Compared to the other views, its maintenance results in the
highest throughput.  The performance depends on the amount of records
selected. Interesting is that the absolute throughput is limited by
the throughput clients exert on the system. In
Figure~\ref{fig:selection}, the performance is monotonically decreasing
for a selectivity of 10\%. This means, VMs propagate updates as fast
as they are applied to HBase by clients. Increasing the number of view
managers only slows down the system as more components are running
concurrently.

Figure~\ref{fig:viewtypes} shows the performance of the aggregation
view types: \texttt{COUNT}, \texttt{SUM}, \texttt{MIN}, \texttt{MAX},
and \texttt{INDEX}. Again, we use a fixed configuration comprised of
10 region servers and 40 clients that generated 1M operations.
Aggregation views derive from an auxiliary view (here a \texttt{DELTA}
view.)  Therefore, the throughput for aggregation views is lower than
for the \texttt{SELECTION} view. However, in contrast to the
\texttt{SELECTION} view, the throughput significantly benefits from
increasing the number of VMs. Not surprisingly, \texttt{COUNT} and
\texttt{SUM} views show similar performance, as their maintenance is
nearly identical.  \texttt{INDEX} view maintenance exhibits a lower
throughput than \texttt{COUNT} and \texttt{SUM}, which only store one
attribute for the aggregated value, whereas the \texttt{INDEX} view
stores a variable number of primary keys associated with the index
column value of the indexed table.  The performance of both
\texttt{MIN} and \texttt{MAX} is worse compared to the \texttt{INDEX}
view. In a \texttt{MIN} (max) view, we also store base table records,
together with the aggregated value.  The storage overhead is equal to
\texttt{INDEX}, but sometimes the values of an entire row needs to be
queried to recalculate the new minimum (maximum).  


Figure~\ref{fig:join} shows the evaluation results for the
\texttt{JOIN} view under the same conditions as above. Compared to the
other view types and not surprising, the \texttt{JOIN} shows lower
throughput.  The \texttt{JOIN} view requires a more complex internal
auxiliary view table constellation and maintenance. However, we suspect
that throughput is still higher than as if full table scans would have
to be used to find matching rows (not considering consistency issues,
if scans would be used.)  Similar to aggregation views, \texttt{JOIN}
benefits from an increased number of view managers. Also, here as
well, auxiliary tables for \texttt{JOIN} can be amortized as more
views are defined.

\begin{figure*}
\minipage{0.32\textwidth}
  \includegraphics[width=\linewidth]{figures/benefit}
      \vspace{-5mm}
  \caption{Benefit of view maintenance}\label{fig:benefits}
        \vspace{-3mm}
\endminipage\hfill
\minipage{0.32\textwidth}
  \includegraphics[width=\linewidth]{figures/scale_rs}
      \vspace{-5mm}
  \caption{Scale region servers}\label{fig:regionservers}
        \vspace{-3mm}
\endminipage\hfill
\minipage{0.32\textwidth}%
  \includegraphics[width=\linewidth]{figures/zipf}
      \vspace{-5mm}
  \caption{Zipf distribution}\label{fig:zipf}
        \vspace{-3mm}
\endminipage
\end{figure*}
\begin{figure*}
\minipage{0.32\textwidth}
  \includegraphics[width=\linewidth]{figures/cost}
      \vspace{-5mm}
  \caption{Cost of view maintenance}\label{fig:cost}
      \vspace{-5mm}
\endminipage\hfill
\minipage{0.32\textwidth}
  \includegraphics[width=\linewidth]{figures/scale_views}
      \vspace{-5mm}
  \caption{Scale views}\label{fig:views}
      \vspace{-5mm}
\endminipage\hfill
\minipage{0.32\textwidth}%
  \includegraphics[width=\linewidth]{figures/killvm}
      \vspace{-5mm}
  \caption{Fault tolerance - Impact of crash}\label{fig:killvm}
      \vspace{-5mm}
\endminipage
\end{figure*}


\subsubsection{Costs and benefit of view maintenance} 

To determine the
benefits, i.e., the latency improvement a client experiences when
accessing a view, and the costs, i.e., essentially, the decrease in
overall system throughput that results, we conducted an experiment
with three different view maintenance strategies: (i) \textit{Client
  table scan:} To obtain the most recent count view values, a client
scans the base table and aggregates all values on its own. (ii)
\textit{Server table scan:} To obtain the most recent count view
values, the client sends a request to HBase, which internally computes
the view in a custom manner and returns it as output. In the
implementation, we use HBase's ability to parallelize table access.
All region servers scan their part of the table and, at the end,
intermediate results are collected and merged.  (iii)
\textit{Materialized view:} Views are maintain incrementally by VMS
configured with 40 VMs used to materialize the count view in parallel.
For (i) and (ii), we disregard inconsistencies that may result from
concurrent table updates while scans are in progress. Our primary
objective is to obtain some understanding of how VMS fares relative to
other potential approaches.

The results are given in Figure~\ref{fig:benefits}, where we measure
the latency a client perceives (i.e., the time to wait for the result)
as the scanned key range increases.  The first strategy performed
worst. \textit{Client table scans} are sequential by nature and
require a large number of RPCs to HBase, even if requests are
batched. Especially with increasing table size, this approach becomes
more and more impracticable.

\textit{Server table scan} is promising because HBase is able to
exploit the data distribution, which results in a linear speed-up.
Nonetheless, the time to obtain the most recent values for a count
aggregate reached 5~s for a table with 1M rows.  While this result
could now be cached in materialized form, in scenarios with frequently
changing base data, recalculations would have to be frequently
repeated to keep the results up-to-date. Moreover, concurrent table
scans may interfere with the write performance of region servers.

Because views are materialized and updates are done incrementally,
latencies for the third strategy are exactly the same as for every
HBase table. Moreover, the client only accesses the aggregated values
and not the base table. Therefore, the latency remains below 1~ms,
even for a base table with 10 million rows.

Materializing views comes at a cost. We assess this cost as the
performance impact on base table creation time (here, defines as the
time to insert 1M tuples into the base table.)  Figure~\ref{fig:cost}
shows the base table creating time with and without concurrent view
maintenance.  We track creation time as the number of clients
increases.  Figure~\ref{fig:cost} shows view maintenance with 20, 30
and 40 VMs.  While view maintenance increases the base table creation
time by approximately 40\%, a further increase of VMs (by ten) only
increases base table creation time by approximately 2\%.


\subsubsection{System scalability}  
In Figure~\ref{fig:selection} and
Figure~\ref{fig:viewtypes}, we evaluated the performance of different
view types. In both experiments, we increased the number of VMs.  Now,
we examine system performance when scaling up region servers. In
Figure~\ref{fig:regionservers}, the number of region servers is varied
from 4 to 10. We run the experiment with a selection, a count and an
index view and measure throughput. We observe an almost linear
increase of throughput independent of the view type processed.

The effect can be explained as follows: Each region server runs on a
separate node. Adding another region server results in additional I/O
channels (i.e., separate disk).  Thus, the overall performance of
HBase increases. Clients requests are completed faster due to the
additional I/O capacity.  Likewise, VMs can perform faster view
updates.  We draw the following conclusion: Scaling up the number of
VMs as in Figure~\ref{fig:selection} and Figure~\ref{fig:viewtypes}
improves the view maintenance throughput up to a point where VMs are
saturated with updates that they can push through the available I/O
channels.  Scaling up the region server improves overall performance,
also for VMs, due to the additional I/O capacity.

Figure~\ref{fig:views} shows system throughput as multiple views are
maintained by a varying number of VMs.  We note a performance leap as
load changes from maintaining one to two count views. In our workload
the count views derive from the same auxiliary table, which must be
maintained as well, but only once. In further increasing the number of
views, the effect diminishes.  All aggregation views, especially join
views, benefit from the sharing of auxiliary views.

However, increasing the number of views in the system, also increases
the lag between base table states and derived view table
states. While, as we show in Section~\ref{sec:consistency}, our view
states are consistent, they do lag behind in time.  This lag can be
reduced by increasing the number of VMs to speed up view maintenance.
Thus, the more views we maintain, the more important is the number of
VMs allocated.

\subsubsection{Impact of data distribution}  

Figure~\ref{fig:zipf}
evaluates the effect of different distributions on system performance.
We scale the number of VMs and track the latency for creating base
tables and derived views.  Keys of update operations are drawn from a
Zipf and a Uniform distribution.

For the Uniform distribution, creation latencies are independent of
the number of VMs assigned, even small numbers of VMs can handle the
load in our set-up.  For the Zipf distribution, latencies are much
higher. We also see that creation and maintenance latency increase,
yet are positively effected by increasing the number of VMs that
propagate updates.  Thus, especially for a skewed workload, the system
greatly benefits from being able to dynamically assign VMs.  VMs can
be assigned to hot spots in the key range, away from ranges where they
are not needed.  Nevertheless, in this case, HBase may constitute a
bottleneck for clients. Clients issue updates to a set number of
region servers that handle 90\% of the load, thus, resulting in a
slowdown.  HBase developers suggest that keys of updates should be
salted. That is a prefix is assigned to keys, such that their
distribution becomes uniform again. In this case, VMs propagate
updates as we saw above.


\subsubsection{Impact of VM crash}  
Figure~\ref{fig:killvm} shows the
impact of a view manager crash on system performance.  In this
experiment, we track view maintenance latency, as the number of VMs
available in the system crashes.  We ran experiments with 20, 30 and
40 VMs, respectively. Twenty seconds after view maintenance started,
we terminate a number of VMs. In this experiment, the VMs are
distributed evenly to region servers. In a set-up with 20 VMs and 10
region servers, we have a ratio of 2 VMs per region
server. Terminating both VMs of the same region server stops view
maintenance. The operations arriving at that region server cannot be
forwarded until a new VM is assigned to the region server. In the
experiment, we terminate VMs of different region servers.

In our set-up, a single crashing VM reduces the view maintenance
latency, because it takes additional time for recovery to replay the
TL and because the remaining VMs have to absorb additional load.  The
failing of further VMs does not further impact the situation.  As long
as every region server loses only one of its two VMs, the overall
processing time remain the same.  Overall, the impact of a VM crash
lessens, the more VMs are deployed.  Thus, increasing the number of
VMs, increases the fault resilience of the system.


%
\section{Related work}
The \textit{proceedings} are the records of a conference.
ACM, as well as PVLDB, seeks to give these conference by-products a uniform,
high-quality appearance.  To do this, ACM / PVLDB has some rigid
requirements for the format of the proceedings documents: there
is a specified format (balanced  double columns), a specified
set of fonts (Arial or Helvetica and Times Roman) in
certain specified sizes (for instance, 9 point for body copy),
a specified live area (18 $\times$ 23.5 cm [7" $\times$ 9.25"]) centered on
the page, specified size of margins (2.54cm [1"] top and
bottom and 1.9cm [.75"] left and right; specified column width
(8.45cm [3.33"]) and gutter size (.083cm [.33"]).

The good news is, with only a handful of manual
settings\footnote{Two of these, the {\texttt{\char'134 numberofauthors}}
and {\texttt{\char'134 alignauthor}} commands, you have
already used; another, {\texttt{\char'134 balancecolumns}}, will
be used in your very last run of \LaTeX\ to ensure
balanced column heights on the last page.}, the \LaTeX\ document
class file handles all of this for you.

The remainder of this document is concerned with showing, in
the context of an ``actual'' document, the \LaTeX\ commands
specifically available for denoting the structure of a
proceedings paper, rather than with giving rigorous descriptions
or explanations of such commands.


%
\section{Conclusions}
\label{sec:conclusion}

In this paper, we developed a scalable view maintenance system, fully 
integrated with a distributed KV-store. We demonstrated the 
efficient, incremental, and deferred materialization of selection, 
index, aggregation, and join views with HBase. Our approach is capable 
to consistently maintain multiple views that may depend on each other. 
In the spirit of KV-stores, our view maintenance architecture is 
incrementally scalable by adding additional view managers as maintenance 
load increases. In our approach, a stream of base table updates is 
propagated to view tables by a bank of view managers operating in 
parallel. To establish view table consistency, we resort to the 
application of a number of known techniques that are combined in novel 
ways to materialize views consistently. We also address fault tolerance 
and recovery to react to failing view managers in our approach. Our 
experimental evaluation quantified the benefits and cost of the approach 
and shows that it scales linearly in view update load and number of view 
managers running. 

In future work, we aim at exploring optimizations for the maintenance of 
multiple, overlapping view expressions and explore automatic means for 
reacting to view maintenance load variations in our architecture. 

% The following two commands are all you need in the
% initial runs of your .tex file to
% produce the bibliography for the citations in your paper.
\bibliographystyle{abbrv}
\bibliography{vldb_sample}  % vldb_sample.bib is the name of the Bibliography in this case
% You must have a proper ".bib" file
%  and remember to run:
% latex bibtex latex latex
% to resolve all references
%
%\subsection{References}
%Generated by bibtex from your ~.bib file.  Run latex,
%then bibtex, then latex twice (to resolve references).

%APPENDIX is optional.
% ****************** APPENDIX **************************************
% Example of an appendix; typically would start on a new page
%pagebreak

%\begin{appendix}
%You can use an appendix for optional proofs or details of your evaluation which are not absolutely necessary to the core understanding of your paper. 
%
%\section{Final Thoughts on Good Layout}
%Please use readable font sizes in the figures and graphs. Avoid tempering with the correct border values, and the spacing (and format) of both text and captions of the PVLDB format (e.g. captions are bold).
%
%At the end, please check for an overall pleasant layout, e.g. by ensuring a readable and logical positioning of any floating figures and tables. Please also check for any line overflows, which are only allowed in extraordinary circumstances (such as wide formulas or URLs where a line wrap would be counterintuitive).
%
%Use the \texttt{balance} package together with a \texttt{\char'134 balance} command at the end of your document to ensure that the last page has balanced (i.e. same length) columns.
%
%\end{appendix}



\end{document}
