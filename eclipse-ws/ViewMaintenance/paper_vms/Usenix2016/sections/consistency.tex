%
% aj - We need to think hard about this.
%       Originally, we had defined these consistency levels on the basis of
%        records; we now seem to talk about entire views. I couldn't quickly
%        convince myself that this is the same.
%
\section{Consistency Model}
\label{sec:consistency}

Since consistency is a major concern during incremental view
maintenance -- additionally, we introduce a high degree of concurrency --
we define a consistency model before building the VMS. 
In data warehouse environments, for batch-based view update
processing, consistency models have been extensively explored (e.g.,
~\cite{zhuge:view, wang:efficient, zhang:parallel, zhuge:strobe}. For
view maintenance in KV-stores a model has been proposed
in~\cite{jacobsen:viewmaintenance}. First, we adapt this model and the
corresponding consistency levels to match the characteristics of 
KV-stores (defined in the previous section). Second, we define a theorem, 
capturing the requirements against our VMS, in order to achieve strong
consistency.

%
%As defined in the previous section, the KV-store provides the VMS with
%multiple streams of client operations in parallel () 
%
%In VMS, base table update operations initiated by clients are streamed 
%from multiple system nodes and are processed in parallel, which 
%apply resulting streams of updates across the entire system. This design 
%was chosen to maximize the processing capabilities for view update processing. 
%Thus, the consistency model is based on the assumption that a sequence of 
%operations, applied to the same row 
%key, is not re-ordered, which is a guarantee given by all the KV-stores 
%we considered. This is a property referred to as \textit{record timeline} 
%semantics~\cite{cooper:pnuts}.

\subsection{Definition} 
\label{subsec:definition} 

A view data consistency model for incremental, deferred view
maintenance explains whether the view state eventually converges,
whether all intermediate view states are valid, and whether the
sequence of view states produced by base table operations corresponds to
the sequence of operations applied to base data~\cite{zhuge:view,
  wang:efficient, zhang:parallel, zhuge:strobe,
  jacobsen:viewmaintenance}. Generally speaking, depending on view
types, view maintenance strategies, and view update programs, none,
some, or all of the listed properties are attainable~\cite{zhuge:view,
  wang:efficient, zhang:parallel, zhuge:strobe,
  jacobsen:viewmaintenance}.




In our context, we stream operations from multiple 
sources (i.e. nodes) and apply them to different parts of the view 
table, located on different nodes. We want to parallelize view 
maintenance and improve performance to a great extent, we further relax 
the consistency model. We do not claim the stream of a local source to 
be in sequence. Instead, we define consistency on a record base level. 
Every sequence of operations, that is applied to a specific row key has 
to be in sequence. A base state $B_j$ is said to be greater than $B_i$ 
if all operations to a specific row key in $B_i$ are included in $B_j$ 
plus a number of subsequent operations. Formally we write $(\exists t_k 
\in B_j)(\forall t_l \in B_i)\;k > l\;\land\; r(t_k)=r(t_l)$. Function 
$r()$ delivers, when applied to an operation, the row key of that 
operation. Now, we define consistency levels in agreement to the 
model. Formally, the consistency levels of the model are 
defined as follows:


\begin{enumerate}
 \setlength{\itemsep}{5pt}
\item[]\textit{Convergence:} A view table converges, if after the system
  quiesces, the last view state $V_f$ is computed correctly. This
  means it corresponds to the evaluation of the view expression over
  the final base state $V_f=View(B_f)$. View convergence is a minimal 
  requirement, as an incorrectly calculated view is of no use.

\item[]\textit{Weak consistency:} Weak consistency is given if the view
  converges and all intermediate view states are valid, meaning that
  there exists a base table state in the sequence of base table states
  produced by the operation sequence from which they can be derived
  %$V_j=View(B_i)$.

\item[]\textit{Strong consistency:} Weak consistency is achieved and the
  following condition is true. All pairs of view states $V_i$ and
  $V_j$ that are in a relation $V_i \leq V_j$ are derived from base
  states $B_i$ and $B_j$ that are also in a relation $B_i \leq B_j$.

\item[]\textit{Complete consistency:} Strong consistency is achieved
  and every base state $B_i$ of a valid base state sequence is
  reflected in a view state $V_i$. Valid base state sequence means
  $B_0 \leq B_i \leq B_{i+1}\leq B_f$.
\end{enumerate}


%\begin{enumerate}
%\item \textit{Convergence}: A view converges, if after the system
%  quiesces, the final view state \textit{corresponds} to the final
%  base state.  By corresponds we mean that the evaluation of the view
%  expression over the base state results in the view state. We say the
%  final view state is correct.
%%  
%\item \textit{Weak consistency} Weak consistency is given, if the view
%  converges and all intermediate view states are valid. Informally,
%  this means that for every view state there exists a base table state
%  which corresponds to the view state. In particular, this means that
%  there never is an invalid view state.
%%
%\item \textit{Strong consistency}: Weak consistency is achieved and
%  and every pair of view states are correctly ordered (i.e.,
%  correspond to two base states in the same order.)
%%
%\item \textit{Complete consistency}: Strong consistency is achieved
%  and every base state is reflected in some view state.
%\end{enumerate}
%%


\subsection{VMS requirements} 
\label{subsec:vms_requirements} 

A system that is to achieve a given consistency level has to offer
certain guarantees based on which the consistent and correct
materialization of views can be based.  We show that our approach can
attain strong consistency for the views it maintains, which we capture
in the following theorem.\\


\newtheorem{theorem}{Theorem}
\begin{theorem}
\label{theo:strong_consistency}
The following requirements are sufficient to guarantee that views are
maintained \textit{strongly} consistent in a VMS.

\begin{enumerate}
	\item  View updates are applied \textit{exactly once}
	\item  view record updates are \textit{atomic}
	\item  record \textit{timeline} is always preserved
%	\item  View records are processed \textit{locally}
\end{enumerate}
\end{theorem}
\hfill \newline
Due to the space constraints, we present the theorem without proof.
We refer the reader to \cite{extended:version} for the complete proof. 
Now, we explain every requirements and its implications in detail. 

\subsubsection{Exactly once property}
\label{subsubsec:exactly_once}  
Updating a view exactly once for every client operation is critical, 
as views can be non-idempotent. There are two possible incidents that 
violate the exactly once requirement: an operation is lost (maybe due to 
node crash, or transmission errors); an operation is applied 
more than once (in consequence of a log replay after a node crash). 
In either case, the view would be incorrect (i.e., does not converge).
 


%The remedy involves log replaying, 
%which may lead to an update being applied more than once.  Transmission 
%errors are prevented by using reliable communication channels (in our 
%cases realized as TCP sockets.)
%
%Update applied more than once} -- If the same update is
%applied more than once to a non-idempotent view, the result is
%incorrect. Here, we adopt the idea of a \textit{signature} to address
%this problem ~\cite{jacobsen:viewmaintenance}.  The signature uniquely
%identifies updates.  Every time a base table update is propagated to a
%view, its signature is evaluated to ensure the update has not already
%been propagated to the view.

%In our implementation we construct a globally unique
%\textit{sequence-ID} for base table operations: Every node orders base
%table operations using a locally unique \textit{sequence number}. When
%an operation is written to the write-ahead log, the sequence number is
%incremented by one. We combine the sequence number with the region
%server ID to obtain a globally unique identifier, which uniquely
%identifies whether a base table operation has already been applied to
%the view record or not.

\subsubsection{Atomic view record update}
\label{subsubsec:atomic_update} 

Most view types define a mapping from multiple base table records to 
a single view table record. Different base table records may be 
propagated to different VMs. When a view is updated it is possible that 
multiple VMs are concurrently updating the same view record The following 
example illustrates this situation.

\noindent
\textit{Example~1:} Given base table $A(\underline{K}, F)$ and a \texttt{SUM} 
view $S(\underline{X}, F)$, defined as $S=\gamma_{c_1,Sum(c_2)}(A)$. 
Assume a client inserts two records into base table $A$. The KV-store writes the following 
operations into the transaction log: $t_1=put(k_1,\{\langle 
c_1,x_1\rangle,\langle c_2,a\rangle\})$ and $t_2=put(k_2, \{\langle 
c_1,x_1 \rangle,\linebreak \langle c_2,b\rangle\})$. Let the VMS receive both 
operations and process them in parallel. To perform view maintenance, 
the VMS retrieves the corresponding old view records from the view 
table. Since $S$ is a \texttt{SUM} view -- and $t_1$ and $t_2$ refer to 
the same aggregation key $x_1$ -- the VMS retrieves the same view record 
twice, e.g. $(x_1, \{\langle c_s, v_s\rangle\})$. In case of $t_1$ VMS 
adds the delta $a$ to the view record; in case of $t_2$ VMS adds the 
delta $b$. Then, the VMS writes the updated view records back to the 
view table. Depending on which record is written first, the VMS 
overwrites one update with the other. Say, VMS writes $(x_1, \{\langle 
c_s, v_s+a \rangle\})$ to $S$; then it overwrites the result with 
$(x_1,\{\langle c_s, v_c+b\rangle\})$. In consequence the delta $a$ is 
missing in the view. The correct result should be $(x_1,\{\langle c_s, 
v_s+a+b \rangle\})$. Therefore, atomic view updates are essential. 



%In our implementation, we use a test-and-set mechanism to solve this
%problem.~\footnote{In HBase a \texttt{checkAndPut} method is provided
%  to realize this mechanism. Most KV-stores offer a similar
%  abstraction.} When updating a table record, a caller sees (tests)
%whether a record has been concurrently modified between a read and an
%update.
%
%Revisiting Example~2, let $VM_2$ retrieve value $(x1, \{(col_1,
%a)\})$. Then, it computes $(x1,\{(col_1,a+c)\})$ trying to put the new
%value, while testing for the old value $a$. The test-and-set fails
%because the old record value changed concurrently to $a+b$. Thus,
%$VM_2$ fetches the updated value again and re-computes
%$(x1,\{(col_1,a+b+c)\})$. This time the test-and-set succeeds and the
%record is written.

\subsubsection{Record timeline}
\label{subsubsec:record_timeline} 

Record timeline means that sequences of operations on the same row key are not re-ordered 
when processed by VMS.  Again, a little example demonstrates the
importance of record timeline semantics.

\noindent
\textit{Example~2:} Given base table $A(\underline{K}, F)$ and a 
\texttt{SELECTION} view $S(\underline{K}, F)$, defined as 
$S=\sigma_{c_2 < x}(A)$. A client 
executes two operations on base table $A$. Operation $t_1$ inserts key 
$k_1$ to the table; operation $t_2$ updates key $k_1$. Since both operations
touch the same key, KV-store writes them into the TL in sequence: 
$t_1=put(k_1,\{\langle c_1,x_1\rangle, \linebreak\langle c_2,a\rangle\})$ and 
$t_2=put(k_1, \{\langle c_1,x_1 \rangle, \langle c_2,b\rangle\})$. Assume 
that both operations match the selection condition (i.e. $(a < x)$ and $(b < x)$) 
such that both operations trigger an update on the view table. Let 
the VMS receive both operations and process them in parallel. Due to  
concurrency, the order of both operations could be reversed. First, the
view record $k_1$ is updated with $t_2$, second with $t_1$. Then final
state of the view record will be  $(k_1, \{\langle c_1,x_1 \rangle,\langle 
c_2,a\rangle\})$ while the base table contains $(k_1, \{\langle c_1,x_1 \rangle,\langle 
c_2,b\rangle\})$. As we observe, breaking the record timeline may also 
violate convergence -- which is a minimal requirement for view maintenance. 


%\subsection{Proof of Consistency}


Theorem~\ref{theo:strong_consistency} states that a VMS system fulfilling all 
three requirements achieves at least strong consistency. In the following, we 
will present a proof for this theorem, which is organized in three stages: we 
start with proving convergence and then present extensions to also prove weak 
consistency  and finally strong consistency.





\subsubsection{Notation}
First, we define the following notation for keys, 
operations on keys, and the ordering of operations. Let $k_x$ denote a key in 
a base table, where $x \in X = \{1, \dots, m\}$, and $X$ is the 
table's key range. Further, let an operation on key $k_x$ be defined as 
$t[k_x]$, and a totally ordered sequence of such operations be denoted by 
$\langle t_1, t_2, t_{3}, \dots, t_N\rangle$, where $N$ defines the total 
number of operations on the table in a given timespan. 
Hence, a generalized sequence of operations on a base table is 
represented by $\langle t_1[k_{x_1}], t_2[k_{x_2}], \dots, 
t_n[k_{x_m}]\rangle$, where $k_{x_1}, \dots, k_{x_m} \in X$ can be arbitrarily 
chosen from the base table's key range for every timestamp. Based on this 
generalized sequence every other sequence of operations can be derived, e.g. 
$\langle t_1[k_{1}],t_2[k_{2}],t_3[k_{1}]\rangle$. In some cases, we also want 
to keep the ordering of operations variable. For that reason, we introduce an 
index $s_i$ with $i \in \{1, \dots, n\}$. Then we can write the arbitrary 
sequence as $\langle t_{s_1}[k_{x_1}], t_{s_2}[k_{x_2}], \dots, 
t_{s_n}[k_{x_n}]\rangle$ with $s_1\neq \dots \neq s_n$. Using this notion, we 
are capable of representing every possible sequence of update operations in 
the system. 

The index $t^{(i)}[k_x]$ is used to express a sequence of operations on 
a single row key (i.e. the time-line). For example, a sequence of 
operations on row key $k_x$ is denoted as $\langle 
t^{(1)}[k_x],t^{(2)}[k_x]...,t^{( \omega)}[k_x]\rangle$. The last 
operation on a particular row key is always denoted with $\omega$. (see 
Section~\ref{sec:consistency}).

For the proof, we assume such an arbitrary sequence of base table operations
 and then we show that --- given the requirements in the theorem --- a VMS 
 system will produce correct view results. Formally, we show that $V_f=View(B_f)$. 

\subsubsection{Convergence}
\label{sub:proof_convergence}
As mentioned earlier, every view type defines its own mapping from base table 
to view table records. Thus, we prove the different cases separately. 

\noindent
\textit{One-to-one mapping:} \texttt{SELECTION}, \texttt{PROJECTION} 
views define a one-to-one mapping between base and view table. The row 
key of the base table is also the row key of the view table. Operations 
for both view types are idempotent, meaning an operation could be 
applied several times without changing the result. The view record is 
always a representation of the last base table operation applied. A 
correct view record with row key $k$ is defined as the last operation on 
the row key in the base table key, e.g. $k\leftarrow 
View(t^{(\omega)}[k])$. The function $View$ calculates the view record 
(by using the appropriate view definition) and applies the result to the 
correct view key. A view table converges, if all view records are 
computed correctly in the last view state. 

We start defining an arbitrary sequence of operations on the base table, 
shown in Step~\ref{proof:oo_step1}. Clients can update different row 
keys in the base table, using put or delete operations. Likewise, they 
can update the same row key multiple times. The update operations of the 
clients form a particular global order, expressed through $t_1..t_n$. 
They take the base table from its initial state $B_0$ to its final state 
$B_f$. In the next step, all update operations get forwarded, causing 
the ordering of operations to be lost. If we would continue working with 
an unordered set, convergence of the view will be violated at some 
point. For this reason, we apply requirement (iii) (,i.e. the time-line 
requirement) to our equation, as depicted in Step~\ref{proof:oo_step3}. 
As updates operations do not influence each other (see requirement 
(ii)), we are able to create a set of sub-sequences. These 
sub-sequences only consist of updates that have been applied to the same 
row key. Likewise the sub-sequences contain all operations from the 
previous step (see requirement (i)). 
%
\begin{subequations}
  \begin{align}
 S_b&=\langle t_1[k_{x_1}],....,t_n[k_{x_n}] \rangle;\;\;\Big(B_0 \overset{S_b}{\rightarrow}B_f\Big) \label{proof:oo_step1}\\ 
 S_1&=\langle t^{(1)}[k_{1}],..,t^{(\omega_1)}[k_{1}]\rangle,..,S_m=\langle t^{(1)}[k_{m}],..,t^{(\omega_m)}[k_{m}]\rangle\label{proof:oo_step3}\\
  S_1&=\langle t^{(\omega_1)}[k_{1}]\rangle,..,S_m=\langle t^{(\omega_m)}[k_{m}]\rangle\label{proof:oo_step4}\\
 S_v&=\langle t_{s_1}^{(\omega_1)}[k_{1}],..t_{s_n}^{(\omega_m)}[k_{m}]\rangle;\; \Big(V_0\overset{S_v}{\rightarrow}V_f\Big)\label{proof:oo_step5}\\
 	V_f&=k_x\leftarrow View(t^{(\omega_x)}[k_x])=View(B_f)\label{proof:oo_step6}
  \end{align}
\end{subequations}
%
In Step~\ref{proof:oo_step4}, we pick the last element of all 
sub-sequences and eliminate the rest. As stated above, only the last 
operation on a particular row key has an influence on the final view 
state. Again, we observe that the time-line of a row key is vital. If it 
is broken, e.g. for row key $k_1$, then an operation 
$t^{(\omega-1)}[k_{1}]$ can be incorporated into the final result and 
render it incorrect. After the elimination, we unite the operations 
again in Step~\ref{proof:oo_step5}. The reunion allows the remaining 
operations to be executed in every possible order (i.e. every operation 
could be executed in parallel). Finally, the view definition is applied 
to every operation in $R$. This leads to the correct final view records 
and hence, to convergence. The \texttt{DELTA} view also defines a 
one-to-one mapping between base and view records. In contrast to the 
aforementioned views, the results of the \texttt{DELTA} view do not only 
relate to the last, but to the two last operations. Therefore, we need 
to change the last three steps of the proof as follows: 
%
\begin{subequations}
  \begin{align}
  S_1&=\langle t^{(\omega_1-1)}[k_{1}],t^{(\omega_1)}[k_{1}]\rangle,..,S_m=\langle t^{(\omega_m-1)}[k_{m}],t^{(\omega_m)}[k_{m}]\rangle\\
   S_v&=\langle t_{s_1}^{(\omega_1-1)}[k_{1}],t_{s_2}^{(\omega_1)}[k_{1}],..t_{s_{n-1}}^{(\omega_m-1)}[k_{m}],t_{s_n}^{(\omega_m)}[k_{m}]\rangle\label{proof:ood_step2}\\
   &(\forall t_{s_i}^{(\omega_y-1)}\in S_v)(\exists t_{s_j}^{(\omega_y)}\in S_v)\;s_i < s_j;\;\Big(V_0\overset{S_v}{\rightarrow}V_f\Big)\notag\\
 	V_f&=k_x\leftarrow View(t^{(\omega_x-1)}[k_x],t^{(\omega_x)}[k_x])=View(B_f)
  \end{align}
\end{subequations}
%
As can be observed in Step~\ref{proof:ood_step2}, the two last 
operations of a time-line are included into the final result. However, 
the sequence can be arbitrarily ordered, but needs to preserve the 
time-line of both last operations (i.e. $\omega-1$ must always precede 
$\omega$). Computing $V_f$ leads to the valid final state and the 
\texttt{DELTA} view converges. 

\noindent
\textit{Many-to-one mapping:} (\texttt{PRE}-)\texttt{AGGREGATION} and 
\texttt{INDEX} views define a many-to-one mapping between base and view 
table. The row key of the view table is the aggregation key. Multiple 
row keys in the base table can relate to a particular aggregation key. 
However, a base table row has always only one aggregation key. A correct 
view record with aggregation key $x$ is defined as the combination of 
multiple base records $k_{x_1}..k_{x_j}$, related to the particular key. 
In terms of incremental view maintenance, the correct view record can be 
defined as a number of last operations, that have been applied to this 
combination of base records: $x \leftarrow 
View(t^{(\omega_1)}[k_{x_1}],..,t^{(\omega_j)}[k_{x_j}])$. In case of a 
\texttt{SUM} view e.g., this resolves to $x \leftarrow 
f(t^{(\omega_1)}[k_1])+..+f(t^{(\omega_j)}[k_j])$. We start again, 
defining an arbitrary sequence of base table operations in 
Step~\ref{proof:mo_step1}. In contrast to the previously handled views, 
we are now processing $\delta$-operations. We construct a number of $m$ 
sub- sequences, each containing the $\delta$-operations of one 
particular base record key. In Step~\ref{proof:mo_step2}, we merge the 
$\delta$-operations together. All $\delta$- operations add up to form 
the last transaction as the end result (i.e. $\delta(t^{(1)}[k_{1}])+
..+\delta(t^{( \omega_1)}[k_{1}])=t^{(\omega_1)}[k_{1}]$). 
%\begin{subequations}
%  \begin{align}
%  \{t_1(k_1),..,t_n(k_1)..t_1(k_n),..,t_n(k_n)\}\\
% \{\langle t_1(k_1),..,t_n(k_1)\rangle,..\langle t_1(k_n),..,t_n(k_n)\rangle\}\\
% \{\langle t_n(k_1)\rangle,..\langle t_n(k_n)\rangle\}\\
% 	x=f(t_n(k_1))+..+f(t_n(k_n))
%  \end{align}
%\end{subequations}
%
\begin{subequations}
  \begin{align}
  S_b&=\langle t_1[k_{x_1}],....,t_n[k_{x_n}] \rangle;\;\Big(B_0 \overset{S_b}{\rightarrow}B_f\Big) \label{proof:mo_step1}\\ 
 S_1&=\langle \delta(t^{(1)}[k_{1}]),.,\delta(t^{(\omega_1)}[k_{1}])\rangle,..,\label{proof:mo_step2}\\
 &\hspace{10 mm}S_m=\langle \delta(t^{(1)}[k_{m}]),.,\delta(t^{(\omega_m)}[k_{m}])\rangle\notag\\
  S_1&=\langle t^{(\omega_1)}[k_{1}]\rangle,..,S_m=\langle t^{(\omega_m)}[k_{m}]\rangle\\
 S_v&=\langle t_{s_1}^{(\omega_1)}[k_{1}],..t_{s_n}^{(\omega_m)}[k_{m}]\rangle;\;\Big(V_0\overset{S_v}{\rightarrow}V_f\Big)\label{proof:mo_step4}\\
 	V_f&=x\leftarrow View(t^{(\omega_{x_1})}[k_{x_1}],.., t^{(\omega_{x_j})}[k_{x_j}])=View(B_f)\label{proof:mo_step5}
 	%&\hspace{10 mm}x_1,..,x_j \in \{1,..,m\};\;x_1\neq,..,\neq x_j \notag
  \end{align}
\end{subequations}
%
Now, we can unite the single sequences as done before. We retrieve a 
final sequence as shown in Step~\ref{proof:mo_step4}. The operations of 
this sequence are then applied to the view --- simultaneously they are 
grouped and stored according to their aggregation key. The final view 
records are calculated correctly, as depicted in 
Step~\ref{proof:mo_step5}, which causes the aggregation view to 
converge. 

\textit{Many-to-many mapping:} (\texttt{REVERSE}-)\texttt{JOIN} views 
define a many-to-many mapping between base and view table. The row key 
of the view table is a composite key of both join tables' row key. 
Multiple records of both base tables form a set of multiple view records 
in the view table. Since the joining of tables takes place in the 
\texttt{REVERSE JOIN} view, we prove convergence only for this view 
type. A \texttt{REVERSE JOIN} view has a structure that is similar to an 
aggregation view. The row key of the \texttt{REVERSE JOIN} view is the 
join key of both tables. All base table records are grouped according to 
this join key. But in contrast to an aggregation view the 
\texttt{REVERSE JOIN} view combines two base tables to create one view 
table. A correct view record with join key $x$ is defined as a 
combination of operations on keys $k_1..k_n$ from join table $A$ and 
operations on keys $l_1..l_p$ from join table $B$. In order to represent 
both keys we introduce an additional variable $z_1,..,z_n \in 
\{k_1,..,k_m, l_1,..,l_p\}$. Then, the correct view record is defined 
as: $x \leftarrow View(t^{(\omega_1)}[z_1], ..,t^{(\omega_j)}[z_j])$. 
We start with a sequence of arbitrary client updates to both base 
tables, as depicted in Step~\ref{proof:mm_step1}. Then, the order of 
updates is lost and the time-line requirement is realized in 
Step~\ref{proof:mm_step2}. 
%
\begin{subequations}
  \begin{align}
  S_b&=\langle t_1[z_1],..,t_n[z_n]\rangle;\;\Big(B_0 \overset{S_b}{\rightarrow}B_f\Big)\label{proof:mm_step1}\\ 
 S_1&=\langle \delta(t^{(1)}[k_{1}]),..,\delta(t^{(\omega_1)}[k_{1}])\rangle,..,S_m,..,S_{m+1},..\label{proof:mm_step2}\\ 
&\hspace{10 mm}S_{m+p}=\langle \delta(t^{(1)}[l_{p}]),..,\delta(t^{(\omega_p)}[l_{p}])\rangle\notag\\
  S_1&=\langle t^{(\omega_{k_1})}[k_{1}]\rangle,.., S_m=\langle t^{(\omega_{k_m})}[k_{m}]\rangle,..,\\
 &\hspace{10 mm}S_{m+1}=\langle t^{(\omega_{l_1})}[l_{1}]\rangle,..,S_{m+p}=\langle t^{(\omega_{l_p})}[l_{p}]\rangle\notag\\
 S_v&=\langle t^{(\omega_{k_1})}[k_{1}],..t^{(\omega_{k_m})}[k_{m}],\\
 &\hspace{10 mm}t^{(\omega_{l_1})}[l_{1}],..,t^{(\omega_{l_p})}[l_{p}] \rangle;\;\Big(V_0\overset{S_v}{\rightarrow}V_f\Big)\label{proof:mm_step3}\\
 	V_f&=x\leftarrow View(t^{(\omega_{z_1})}[z_1],.., t^{(\omega_{z_j})}[z_j])=View(B_f)\label{proof:mm_step4}
 	%&\hspace{10 mm}z_1,..,z_j \in \{k_1,..,k_m,l_1,..,l_p\}; z_1\neq,..,\neq z_j ;\; \notag
  \end{align}
\end{subequations}
%
We eliminate all but the last operations $\omega$ and reunite the 
operations in Step~\ref{proof:mm_step3}. This leads to the final 
Step~\ref{proof:mm_step4}, where the operations are applied to the view 
record. Since the view records are calculated correctly (i.e. only the 
last operations of the row keys are included) we conclude that the view 
converges. 

\subsubsection{Weak consistency} 
\label{sub:proof_weak}

Weak consistency has been defined as follows: Weak consistency is given 
if the view converges and all intermediary view states are valid, 
meaning they can be derived from one of the base states with 
$V_j=View(B_i)$. As we already proved convergence, we need show that all 
the intermediary view states are correct likewise. We start again with 
an arbitrary sequence of operations in Step~\ref{proofw:oo_step1}. In 
order to generate an intermediate base state, we cut the sequence at any 
point before an operation $t_a$, with $1 < a < n$. After the ordering is 
lost, we apply the time-line consistency. But in contrast to before, we 
are not capable of processing the complete time-line (i.e. 
$(1)..(\omega)$). Instead, we process the time-line until an 
intermediary element $\alpha_x \leq \omega_x$. 
%
\begin{subequations}
  \begin{align}
  S_b&=\langle t_1[k_{x_1}],....,t_n[k_{x_n}] \rangle;\;\Big(B_0 \overset{S_b}{\rightarrow}B_{a}\Big)\label{proofw:oo_step1}\\
 S_1&=\langle t^{(1)}[k_{1}],..,t^{(\alpha_1)}[k_{1}]\rangle,..,S_m=\langle t^{(1)}[k_{m}],..,t^{(\alpha_m)}[k_{m}]\rangle\\
 S_v&=\langle t_{s_1}^{(1)}[k_{1}],..t_{s_i}^{(\alpha_1)}[k_{1}],..,t_{s_j}^{(1)}[k_{m}],..,t_{s_a}^{(\alpha_m)}[k_{m}]\rangle;\;\Big(V_0\overset{S_v}{\rightarrow}V_a\Big)\\
 (\forall& t_{s_1}^{(\lambda_1)}[k_{x_1}] \in S_v)(\forall t_{s_2}^{(\lambda_2)}[k_{x_2}] \in S_v):(x_1=x_2)\;\land\;(\lambda_1<\lambda_2) \Rightarrow s_1 < s_2\notag\\
V_{a}&=k_x\leftarrow View(t^{(\alpha_x)}[k_x])=View(B_{\alpha})
  \end{align}
\end{subequations}


\subsubsection{Strong consistency}
\label{sub:proof_strong}

Strong consistency has been defined as follows: Weak consistency is 
achieved and the following conditions hold true. All pairs of view 
states $V_i$ and $V_j$ that are in a relation $V_i \leq V_j$ are derived 
from base states $Bi$ and $B_j$ that are also in a relation $B_i \leq 
B_j$. Since weak consistency is already proven, we only need to prove 
the statement $V_i \leq V_j \Rightarrow B_i \leq B_j$. If this statement 
is negated, then only two of the following cases can occur: Either $V_i 
\leq V_j \Rightarrow B_i \parallel B_j$ or $V_i \leq V_j \Rightarrow B_i 
\geq B_j$. Both cases can only be constructed by breaking the record 
time-line. To be precise: At least one record has to exists, whose 
time-line is broken. Formally, we demand $ (\exists t_l \in B_i)(\forall 
t_k \in B_j):(r(t_l)=r(t_k))\land(l > k) $. Because requirement (iv) 
prevents the breaking of time-lines, we conclude that both cases are not 
possible. Thus, we have proven strong consistency by contradiction. 


%
%\subsubsection{Local processing}
%
%Record timeline means that updates of the same row key are not re-ordered 
%when processed by the system. 


%We consider the following scenarios. Execution of two operations on
%the same row key, $k_1$ and two operations on different row keys,
%$k_1$ and $k_2$. We distinguish between one client $c_1$ sending both
%operations in sequence versus two clients $c_1$ and $c_2$ concurrently
%sending one each. This gives rise to four cases, covering all possible
%input orderings of two operations (i.e., possible base table update
%sequences). Let the base table be defined as $A(\underline{K},\{F\})$.
%
%\noindent
%Client $\boldsymbol{c_1}$ updates $\boldsymbol{k_1}$,
%$\boldsymbol{k_1}$ (1): The client performs two operations
%$t_1=put(A(k_1,\allowbreak\{\langle c_1,v_1\rangle\}))$ followed by
%$t_2=put(A(k_1,\{\langle c_2,v_2\rangle\}))$ to the same base table
%record. Client operations via the KV-API are synchronous and happen in
%order. For that reason, the node writes $t_1$ to its TL followed by
%$t_2$. Given that the NX reads the TL sequentially, it processes both
%operations in order. For each $t_i$ ($i = 1, 2$), NX selects a VM$_1$
%by computing the hash $h(k_i)$, where it queues the operation.  As
%assumed, the row keys of $t_1$ and $t_2$ are the same, so for both
%operations the same VM is selected, where the view updates are
%processed in order by calling on the synchronous KV-API for view table
%updates.
%
% 
%
%\noindent
%Clients $\boldsymbol{c_1}$, $\boldsymbol{c_2}$ update
%$\boldsymbol{k_1}$ (2): Let client $c_1$ perform $t_1$ and client
%$c_2$ perform $t_2$ (both operations on the same row key). Both
%clients connect to the KV-API and are routed to the same node. Since
%the KV-store executes put operations atomically, and locks a record
%during the operation, one client operation always precedes the
%other. Therefore, the order of $t_1$ and $t_2$ is determined by
%whichever operation happens to be processed first, leading to its
%insertion into the TL. When retrieving $t_1$ and $t_2$, the NX
%computes the hash of the corresponding keys, as above.  Thus, the same
%VM is selected in both cases. Therefore, the order of two updates, on
%the same record, by different clients, is determined by the KV-store;
%subsequently, the order is not changed by VMS.
%
%Based on this analysis, we conclude that VMS maintains record timeline
%consistency for view updates. However, we analysed timeline 
%consistency with regard to a static set-up of view managers. We 
%assumed that the number of VMs per node remains fix during the 
%maintenance process. If we introduce load balancing mechanisms or 
%add new view managers to the system on-the-fly, then we need to review 
%the guarantee of timeline consistency. This is done in the appendix, 
%Section~\ref{sec:dynamic_scalability}.
%
%\begin{algorithm}
%\caption{Operation processing at VM}
%\label{alg:processing}
%\begin{algorithmic}[5]
%\Procedure{$processOperation$}{$t \in T$}%\Comment{$t$ received from NX}
%\State $bt \leftarrow getBaseTable(t)$%\Comment{lookup meta table}	
%\State $V \leftarrow resolveViewDefinitions(bt)$%\Comment{lookup meta table}		
%\ForAll{$v \in V$}
%\While{$\neg succeed$}	
%\State $vt \leftarrow getViewTable(v)$
%\State $vk \leftarrow getViewKey(t, v)$	%\Comment{extract view key}	
%\State $vr' \leftarrow KVStore.get(vt, vk)$
%\State $sig \leftarrow getSignature(t)$%\Comment{retrieve signature}
%\If{$vr' \neq null\wedge hasProcessed(vr', sig)$}
%\State $break$%\Comment{leave if $t$ already processed}	
%\EndIf
%\State $vr \leftarrow computeUpdate(vr', t)$\label{line:compute_update}
%\State $suceed \leftarrow KVStore.checkAndPut(vt, vk, vr', vr)$\label{line:check_and_put}
%\EndWhile %\Comment{retry until view updated}
%\State $writeCommitLog(vt)$
%\EndFor
%\EndProcedure
%\end{algorithmic}
%\end{algorithm}
%%
%
%Algorithm~\ref{alg:processing} shows the general steps a VM executes as
%it propagates updates.  It includes the test-and-set and the signature
%mechanism described above.  The former is realized via the
%\texttt{checkAndPut} method available in HBase, for example.  Thus,
%Algorithm~\ref{alg:processing} enforces the exactly once and the
%atomic update propagation requirements. However, preservance of view record
%timeline semantic is enforced by the design of VMS.



